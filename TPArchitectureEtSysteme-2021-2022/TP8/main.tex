\documentclass[11pt]{article}
\usepackage[utf8]{inputenc}
\usepackage[T1]{fontenc}
\usepackage[francais]{babel}
\usepackage[francais]{layout}
\usepackage{hyperref}
\selectlanguage{french}
\usepackage[toc,page]{appendix}
\usepackage{amsmath}

%\usepackage[dvipsnames]{xcolor}


% NE PAS CHANGER !!
\ifx \public \undefined \def\public{etudiants} \fi
%\ifx \public \undefined \def\public{enseignants} \fi
\usepackage[\public]{../tps}
\usepackage{tikz}

\usepackage{hyperref}
\hypersetup{
    colorlinks=true,
    linkcolor=blue,
    filecolor=magenta,      
    urlcolor=cyan,
}

\urlstyle{same}



% Numéro du TP
\newcommand{\numtd}{03}
% Titre du TP
\newcommand{\titretd}{Input/Output Redirection and Pipes}
\def\tup#1{\langle #1\rangle}
\begin{document}
	
	\entete{\numtd}{\titretd}

\section{Review}
In order to run shell commands on C, we use the \verb*|execlp| command. Download "execlp" to see how this function works.

\begin{itemize}
	\item Add the following line after the  \verb*|execlp| command.
	\begin{verbatim}
		printf("Execlp successfully executed!\n");
	\end{verbatim}
	What do you observe? Why does it happen? Check the manual to see if you can figure out.
	\item Use \verb*|fork| in order to correct this code. Hint: you will also need to use \verb*|wait|.
	\item Now, finally, write into the file \verb*|"pingReader.txt"|  the input from the \verb*|execlp|. Hint: Use the \verb*|dup2| command you saw in class.
\end{itemize}

\section{Shady files}

Download the file \texttt{https://amritasuresh.github.io/teaching/TP8/obsf} from the shell into an empty directory and run it. Two (empty) files should have appeared in your current directory. Delete them using only the command line.

Note: how are files identified in general?

\section{Wrong program}
Consider \verb*|closed_pipe.c|. The son is supposed to print the characters sent to him by the parent. Explain the error you see, and correct the program.

\section{Pipes and code replacement}
\begin{enumerate}
	\item Last week, we used exit signals to communicate between parent and child processes. Modify the code for \texttt{simple.c} from last week to use pipes instead.
	
	\item Write a program (in C) that downloads the archive \verb*|bootstrap.tar.gz| (which can be found in \texttt{https://amritasuresh.github.io/teaching/bootstrap.tar.gz}), and unzips it without creating a temporary file. In other words, we want the command \texttt{curl <url> | tar xz} in C. The curl and tar programs will be called by $\verb*|execlp|$.
\end{enumerate}

\section{Buffered and Unbuffered output}
Download the file \verb*|buffered.c|. What do you expect the output to be when you have a look? What is the actual output. Can you explain the discrepancy?

\section{The function of Hénon}

We will calculate the orbit of a dynamic system of dimension 2. The function of Hénon is described by the system 
\begin{equation*}
	\label{eq:1}
	H_{a,b} = \left\{
	\begin{aligned}
		x_{n+1} &= a - by_{n} - x_{n}^{2} \\
		y_{n+1} &= x_{n}.
	\end{aligned}
	\right.
\end{equation*}
We will use one process to calculate the sequence \( (x_{n})_{n} \)  and another process to calculate the sequence  \( (y_{n})_{n} \).The processes will exchange their data via one (or more) pipe (s). 

\begin{enumerate}
	\item First, in order to avoid having to set up a synchronization between the processes, we can use pauses \verb|sleep(1)|.
	\item How to set up synchronization by signals?
	\medskip
	
Subsequently, we will also create a process dedicated to the exit: this process must write lines in the form \texttt{0.3415 1.2451} where the first number
is $x_{n}$ and the second $y_{n}$ in a file \texttt{henon.dat}.

We can plot the function with the command \texttt{gnuplot henon.p} after having downloaded the script "hennon.p". The file \texttt{henon.dat} must be in the same folder as \texttt{henon.p}. 

Observe the graph you get for values \( a = 1.4 \) et \( b = -0.3 \).
\end{enumerate}


\end{document}
