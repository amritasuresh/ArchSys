\documentclass[11pt]{article}
\usepackage[utf8]{inputenc}
\usepackage[T1]{fontenc}
\usepackage[francais]{babel}
\usepackage[francais]{layout}
\selectlanguage{french}

% NE PAS CHANGER !!
\ifx \public \undefined \def\public{etudiants} \fi
\usepackage[\public]{tps}

% Numéro du TP
\newcommand{\numtd}{06}
% Titre du TP
\newcommand{\titretd}{Construction d'un système d'exploitation Debian Stretch minimal}

\graphicspath{{imgs/}}



\begin{document}

\entete{\numtd}{\titretd}

\begin{introduction}
L'objectif de cette séance de travaux pratiques est de construire un système
Debian Stretch minimal (Stretch, ou 9.X est stable,
<<\url{http://www.debian.org}>>). Cette construction illustre, au travers de
Debian, le principe de démarrage d'un système Unix, son architecture et les
principes de son installation, le partitionnement d'un disque dur, la création
de systèmes de fichiers (formatage), leurs montages, etc.
\end{introduction}

\section{Création d'une machine virtuelle}

L'installation du système d'exploitation (OS) GNU/Linux Debian sera effectuée
dans une machine virtuelle KVM/Qemu disponible sur les machines de la salle de
TP, mais peut également être réalisé sur Oracle Virtual Box
(\url{https://www.virtualbox.org/}).

En très simplifié, une machine virtuelle de ce type peut être résumée par un
disque dur virtuel, un fichier, que nous utiliserons avec l'émulateur
d'ordinateur QEMU.

La première étape consiste à créer le fichier qui nous servira de disque dur
virtuel via la commande <<qemu-img>>.

Utilisez l'aide de <<qemu-img>>, pour générer un disque <<hda.qcow2>> de 5G au
format qcow2 directement dans <</tmp/>>.

\begin{solution}
fhh@melusine ~ \$ qemu-img create -f qcow2 /tmp/hda.qcow2 5G\\
Formatting '/tmp/hda.qcow2', fmt=qcow2 size=5368709120 encryption=off cluster\_size=65536 lazy\_refcounts=off refcount\_bits=16
\end{solution}

\section{Récupération du CD-Rom d'installation de Debian}

Le CD-Rom \emph{<<NetInstall>>} d'installation de Debian ou un CD-Rom
\emph{<<Live>>} de \emph{<<Debian Like>>} est suffisant pour l'installation
minimale du système.

Téléchargez le CD-Rom \emph{"NetInstall"} \textbf{du jour} de Debian Linux ou
n'importe quel Live basé sur Debian.

\emph{NB : La seule condition pour réaliser l'installation est que l'utilitaire
<<debootstrap>> soit présent sur le live CD.}

\emph{NB : La suite de ce TD se base sur le Live CD NetInstall de Debian.}

\section{Démarrer la machine virtuelle sur le CD-Rom}

La commande permettant de démarrer une machine virtuelle 32bits est
<<qemu-system-i386>>.

Consultez la page de manuel de <<qemu>> et trouvez la commande permettant de
démarrer une VM 64bits.

\begin{solution}
qemu-system-x86\_64
\end{solution}

Voici quelques options nécessaires au démarrage de votre VM :

\begin{itemize}
 \item \textbf{-cdrom} spécifie le CDROM à utiliser ;
 \item \textbf{-net} permet de rediriger le port de la machine virtuelle sur
un port de la machine locale ;
 \item \textbf{-m} pour spécifier la quantité de mémoire à utiliser ;
 \item \textbf{-boot} spécifie le média sur lequel démarrer ;
 \item \textbf{-enable-kvm} active l'accélération matérielle.
\end{itemize}

Recherchez les options proposées dans les pages de manuel de <<qemu>> et
démarrez une machine virtuelle 64bits, bénéficiant de l'accélération matérielle,
équipée de 4G de RAM sur l'image de CDROM téléchargée et utilisant le disque
généré précédement.

\begin{solution}
qemu-system-x86\_64 -m 4096 -net user,hostfwd=tcp::2222-:22 -net nic -cdrom downloads/mini.iso -boot
d /tmp/hda.qcow2
\end{solution}

Démarrez la machine virtuelle est choisir \emph{<<Advanced options $>$>>} puis
\emph{<<Rescue mode>>}.

\section{Configuration du système live Debian NetInstall}

Configurez le système Debian NetInstall en suivant les menus proposés et entrez
dans le mode <<restauration>>.

À ce stade et quelque soit le système \emph{<<live>>} choisi, vous devez avoir
accès à un shell.

Vérifiez que vous disposez à minima de la commande \emph{<<fdisk>>} et de
\emph{<<debootstrap>>}. Dans le cas contraire, choisissez un autre Live CD.

\section{Installation du système de base}

\subsection{Partitionnement du disque de la VM}

Partitionnez la machine virtuelle en utilisant l'utilitaire de votre choix
(<<fdisk>> est recommandé).

\begin{solution}
Voir <<fdisk>> avec eux.
\end{solution}

Créez le partitionnement suivant :

\begin{itemize}
 \item première partition primaire : / (type 83 dans fdisk) 4G ext4
 \item première partition logique : swap (type 82) environ 1G
 \item première partition primaire bootable (option 'a' dans fdisk)
\end{itemize}

\subsection{Création des systèmes de fichiers}

Créez les systèmes de fichiers via les commandes \emph{<<mkfs.ext4>>} et
\emph{<<mkswap>>}.

\subsection{Montage des partitions}

Le montage des partitions consiste à rendre accessible un système de fichier
via un répertoire.

Dans cet exemple, seule la racine sera montée. Créez un répertoire
\emph{<<target>>} à la racine de votre live CD est montez la racine sous ce
répertoire.

Pour rappel, les périphériques de la machines sont accessibles dans le
répertoire \emph{<</dev>>}. Le premier disque dur est nommé \emph{<<sda>>}. La
première partition primaire porte le numéro \emph{<<1>>}. Les montages sont
fait via la commande \emph{<<mount>>}.

\subsection{Écriture du système de base}

L'utilitaire \emph{<<debootstrap>>} permet d'installer au travers du réseau le
système de base.

Regardez la syntaxe de cette commande et utilisez-la pour déployer la base de
votre système en utilisant une variante <<minbase>>.

\emph{NB : Utilisez << http://ftp.fr.debian.org/debian >> comme mirroir}

\subsection{Entrée dans le futur environnement}

Cette étape permet d'utiliser le nouveau système fraîchement installé pour y
installer les éléments nécessaires à son fonctionnement.

Le système installé utilisera donc le noyau du live cd déjà chargé.

Pour accéder à toutes les fonctionnalités du système, un certain nombre d'accès
doivent y être ajoutés :

\begin{itemize}
 \item montez /dev pour pouvoir accéder aux périphériques
 \item montez /proc pour pouvoir agir sur les processus et interagir avec le
noyau.
\end{itemize}

\begin{verbatim}
 # mount -t proc none /target/proc
 # mount -o bind /dev /target/dev
 # mount -t sysfs none /target/sys
\end{verbatim}

La commande \emph{<<chroot>>} permettra d'exploiter votre futur environnement.
Regardez sa syntaxe et utilisez-la pour exécuter \emph{<</bin/bash>>} dans
votre futur environnement.

\subsection{Mise à jour du système}

Sous Debian les logiciels sont gérés par la commande \emph{<<apt-get>>}. Par
exemple la commande :

\begin{verbatim}
 # apt-get update
\end{verbatim}

met à jour la liste des paquets (programmes) disponibles.

\begin{verbatim}
 # apt-get upgrade
\end{verbatim}

met à jour les paquets dont une nouvelle version a été trouvée dans la liste
des paquets disponibles.

Pour rechercher un paquet, \emph{<<apt-cache>>} sera utilisé :

\begin{verbatim}
 # apt-cache search motclé
\end{verbatim}

Dans notre cas, aucun paquet n'est à mettre à jour. Pourquoi ?

\subsection{Installation d'un noyau}

Votre futur système à besoin d'un noyau pour fonctionner. Documentez et
utilisez <<apt-get>> pour installer le noyau nommé \emph{<<linux-image-amd64>>}
(il s'agit également du nom du paquet).

\subsection{Installation du chargeur de démarrage Grub}

De la même manière que pour le noyau, installez grub2 sur votre système.
Installez-le dans le MBR.

\subsection{Finalisation du futur environnement}

\subsubsection{Initialisation du mot de passe root}

Choisir un mot de passe administrateur et initialisez le avec la commande
\emph{<<passwd>>}.

\subsubsection{Renseignement des points de montages}

Les systèmes de fichiers à monter au démarrage de votre futur système sont
notés dans \emph{<</etc/fstab>>}.

L'exemple suivant correspond à un fichier \emph{fstab} :

\begin{verbatim}
<file system> <dir> <type> <options> <dump> <pass>
/dev/sda1 /		ext2 defaults 0 0
/dev/sda2 none	swap defaults 0 0
\end{verbatim}

Adaptez cette configuration et inscrivez-la dans votre fichier fstab.

\begin{solution}
Aborder le point des UUID
\end{solution}

\subsubsection{Installation de paquets additionnels}

Pour se connecter au réseau, votre futur système nécessitera un client DHCP.
Cherchez un client DHCP dans la liste des paquets disponible et installez-le.

Installez le package <<systemd>> et creez un lien symbolique <</sbin/init>> pointant vers <</bin/systemd>>.

\subsection{Redémarrage}

Quittez l'environnement \emph{<<chrooté>>} et redémarrez votre système.

\section{Test et évolution de votre VM}

Connectez-vous sur votre machine virtuelle. Quels sont les premiers constats ?

Que retourne la commande \emph{<<date>>}?

Corrigez cette machine pour une utilisation plus simple.



\end{document}
