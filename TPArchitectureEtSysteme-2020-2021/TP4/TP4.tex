\documentclass[11pt]{article}
\usepackage[utf8]{inputenc}
\usepackage[T1]{fontenc}
\usepackage[francais]{babel}
\usepackage[francais]{layout}
\usepackage{hyperref}
\selectlanguage{french}
\usepackage[toc,page]{appendix}


\usepackage[dvipsnames]{xcolor}


% NE PAS CHANGER !!
\ifx \public \undefined \def\public{etudiants} \fi
\usepackage[\public]{tps}
\usepackage{tikz}

\usepackage{hyperref}
\hypersetup{
    colorlinks=true,
    linkcolor=blue,
    filecolor=magenta,      
    urlcolor=cyan,
}

\urlstyle{same}



% Numéro du TP
\newcommand{\numtd}{03}
% Titre du TP
\newcommand{\titretd}{Data Representation}
\def\tup#1{\langle #1\rangle}
\begin{document}
	
	\entete{\numtd}{\titretd}


\subsection*{Warming up:}
For the 16-bit codes: $0000 0000 0010 1010$ and $1000 0000 0010 1010$ Give their values, if they are representing:

\begin{itemize}
    \item a 16-bit unsigned integer;
    \item a 16-bit signed integer;
    \item two 8-bit unsigned integers;
    \item two 8-bit signed integers;
    \item a 16-bit Unicode characters;
    \item two 8-bit ISO-8859-1 characters
\end{itemize}

\bigskip

How would you represent "Hello, how are you?" in ASCII? (look for the comma, question mark, and space characters in the ASCII table)


\section{Character encoding in HTML}
HTML pages have several ways to encode characters:
\begin{itemize}
    \item Entity references which is an alternative name for a series of characters. You can use an entity in the \texttt{\&name;} format, where name is the name of the entity
    \item Numeric references that give the code of a Unicode character, in the form 
    \texttt{\&\#}\emph{nnn}\texttt{;} (décimal) or
\texttt{\&\#x}\emph{nnn}\texttt{;}.\smallskip
    \item Direct binary coding in one of the formats discussed in the course (UTF-8, ISO-8859-1, \ldots). This requires adding a tag of the form
    \begin{quote}
  \texttt{<meta http-equiv="Content-Type" content="text/html; charset=iso-8859-1" />}
 \end{quote}
 in the \texttt{header} of the HTML document.
\end{itemize}
We will consider the names of the following European municipalities:
 \begin{quote}
 Crèvec\oe{}ur (France)\\ L'Ha\"y-les-Roses (France)\\ Krom\v{e}\v{r}\'i\v{z}
(Tchéquie)\\ G\"od\"oll\H{o} (Hongrie)\\ S\"u\ss{}en (Allemagne)\\
Pr\ae{}st\o{} (Danemark)
 \end{quote}

\begin{enumerate}
    \item Create an HTML document that displays these names with their entity references.
    \item Look for Unicode codes for non-ASCII characters and use them to recreate the list with numeric references.
    \item Create an HTML document in UTF-8 directly with your text editor.
\end{enumerate}




\section{Analyzing an error}


\subsection{In the html pages}
Visit the \url{http://www.lsv.fr/~fhh/tp05-1}\bigskip. Describe what is happening using the tables of characters available on the internet (\url{https://fr.wikipedia.org/wiki/Table_des_caractères_Unicode_(0000-0FFF)}). 
And Provide a solution to correct this problem.


\begin{solution}

Nous sommes face à un problème d'encodage. Les caractères spéciaux s'affichent
sur deux ou trois caractères au lieu d'un. Il semble que nous soyons face à une
page UTF-8 interprétée en ISO8859-1.

\begin{itemize}
\item é : à = 00c3 soit 11000011 ; © = 00a8 soit 10101000
\item ç :
\item Ã\"{ } : 
\end{itemize}

\end{solution}



\section{File encoding problem}
Download the text file \url{http://www.lsv.fr/~fhh/tp05-2.txt} and view it in your browser.

Search "é" with Firefox and Chrome.

Search "é" with the command "grep" on the downloaded file the. 

What do you see? 

Suggest a solution to correct this type of problem.


\begin{solution}
Les <<é>> sont codés de deux manière différentes. <<hexdump -C>> donne les codages des caractères.

Solution utiliser uconv -x NFD avant de mettre ce type de fichier dans une base de donnée.
\end{solution}


\section{Floating point}

For this question, you can complete the float-skel.c  and  float-full.c files (look at \url{https://sites.google.com/site/farzadjafarrahmani/home/architecture-systems-course}).


Reminder : In c, the type \texttt{float} is represented according to the IEEE 754 standard in the 32 bit. 1 bit for the sign, 8 for the exponent and 23 for the mantisse. 

\begin{quote}
\verb+typedef struct { int signe; int exposant; int mantisse; } fc;+
\end{quote}


\begin{enumerate}
\item Write a C function that decomposes a float into its three components. For example, the representation of  2.5 in IEEE 754 is:
\begin{quote}
\verb+0 . 1000 0000 . 010 0000 0000 0000 0000 0000+
\end{quote}
In this case, the returned structure would contain $\mathtt{sign}=0$, $\mathtt{exposant}=0x80=128$ and $\mathtt{mantisse}=0x200000=2097152$.

Reminder : To do this, it is convenient to use typecast. 


\item Create a function that does the opposite, that is to say that returns the \texttt{float} corresponding to a given \verb+fc+ structure.


\item Realize the actual addition based on the addition of integers by going through the structures of \verb+fc+. To simplify, we will make the following restrictions: (i) both operands are positive
  (ii) no special cases NaN / Inf etc.

Addition in the \verb+fc+ type is done in three steps:
\begin{enumerate}
\item Standardize the two values, i.e. if the two exponents are
different, we adjust the mantissa of one of the two according to the
difference.
\item Add the sum of the two mantissas, taking into account the  ``hidden" bit
representing the 1.
\item Normalize the mantissa for whatever is in $[1,2)$, while adjusting
the exponent of the result.
\end{enumerate}

\end{enumerate}




\end{document}
