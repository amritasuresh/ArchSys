\documentclass[11pt]{article}
\usepackage[utf8]{inputenc}
\usepackage[T1]{fontenc}
\usepackage[francais]{babel}
\usepackage[francais]{layout}
\selectlanguage{french}

% NE PAS CHANGER !!
\ifx \public \undefined \def\public{etudiants} \fi
\usepackage[\public]{tps}

% Numéro du TP
\newcommand{\numtd}{05}
% Titre du TP
\newcommand{\titretd}{Traîtement des chaines de caractères}

\graphicspath{{imgs/}}

\begin{document}

\entete{\numtd}{\titretd}

\begin{introduction}
 Page web du cours :
 \begin{center}
  \url{http://www.lsv.ens-cachan.fr/~schwoon/enseignement/systemes/ws1617/}
 \end{center}
 Suite aux questions concernant l'exercice d'envois de courriels proposé
 la semaine dernière, nous reviendrons rapidement dans cette séance sur
 l'automatisation d'envois de courriels. Nous passerons ensuite au traitement
 des chaines de caractères.
\end{introduction}

% {{{ Scripts d'envois de Mails

\section{Scripts d'envois de courriels}

Dans la séance précédente, nous vous proposions d'utiliser la commande
<<telnet>> pour envoyer des courriels directement en contactant le serveur SMTP
de votre provider. Deux problèmes ont été signalés :\\

\begin{itemize}
 \item Le serveur SMTP refuse la connexion : ce problème est dû au fait que
vous n'utilisiez pas le bon serveur SMTP. En effet, \textbf{vous devez utiliser
le serveur SMTP de votre provider}. Pour des raisons évidentes de sécurité,
vous ne pouvez pas utiliser, sans authentification, le serveur de mail sortant
d'Orange si vous êtes connecté à internet via un autre prestataire. Vous ne
pouvez pas plus utiliser le serveur SMTP de laposte.net sans authentification.\\
 \item Le mail est refusé ou considéré comme SPAM : pour lutter contre le SPAM,
les providers contrôlent un certain nombre de paramètres des mails tel que la
présence d'un sujet. Les mails que vous envoyez via <<telnet>> peuvent contenir
ces <<headers>>, mais dans l'exemple proposé la semaine dernière, la
modification des headers n'était pas spécifiée.\\
\end{itemize}

Une solution simple pour composer un courriel valide est d'utiliser un outil en
ligne de commande tel que <<s-nail>>, <<heirloom-mailx>>, etc, qui se chargera de
composer un courriel valide.\\

Sur les machines de la salle 411, l'outil installé est heirloom-mailx. Vous
pouvez consulter la page de manuel de l'outil pour découvrir les options
disponibles.\\

La première étape consiste à configurer Heirloom dans le fichier
<<\textasciitilde{}/.mailrc>> pour qu'il utilise le serveur SMTP de votre
choix. Étudiez la configuration suivante et adaptez la à votre configuration
ENSC (serveur ENSPS en smtps, authentification plain) :

\begin{lstlisting}
account gmail {
 set smtp-use-starttls
 set ssl-verify=ignore
 set smtp-auth=login
 set smtp=smtp://smtp.gmail.com:587
 set from="Jon Doe"
 set smtp-auth-user=jon.doe@gmail.com
 set smtp-auth-password=somePasword
}
\end{lstlisting}

\begin{solution}
\begin{lstlisting}
account ensc {
 set ssl-verify=ignore
 set smtp-auth=plain
 set smtp=smtps://smtp.ens-paris-saclay.fr:465
 set from="Francis HULIN-HUBARD <fhh@votre domaine>"
 set smtp-auth-user=<login ensc>
 set smtp-auth-password=XXXXXXXXXXX
}
\end{lstlisting}
\end{solution}

Essayez votre configuration.

\begin{solution}
date +\%s | mail -v -A ensc -s Paf2 fhh@lsv.fr
\end{solution}

Créez un profil utilisant le serveur du département.

\begin{solution}
\begin{lstlisting}
account dpt {
 set smtp=smtp://mailhost.dptinfo.ens-cachan.fr
}
\end{lstlisting}
\end{solution}

Proposez un profil utilisant un serveur Orange, OVH ou Free.\\

L'option <<-A>> permet de sélectionner le profil utilisé pour l'envoi d'un
courriel. 

\begin{lstlisting}
mail -A gmail adresse@domain.tld
\end{lstlisting}

Envoyez un courriel en utilisant votre profil ENSC puis en utilisant le
profil de votre provider.\\

\begin{center}
 \textbf{<<Oui, mais du coup, les mails n'ont rien d'anonyme !!>>}
\end{center}

En effet, par défaut, les mails n'ont rien d'anonyme, mais en étudiant les
paramètres de la commande <<mail>>, vous verrez qu'il est possible de modifier
l'adresse de l'expéditeur du message.\\

Envoyez vous un courriel provenant de l'adresse "jean.bon@cadsi.fr". Étudiez
les entêtes du courriel.

\begin{solution}

Option '-r adresse expéditeur'

cat /import/fhh/.mailrc | mail -r "Jean BON <jean.bon@cadsi.fr>" -s "Fichier de configuration ~/.mailrc" fhh@lsv.fr

Entêtes :

\begin{verbatim}
Return-Path: <fhh@e-galaxie.org>
X-Original-To: fhh@e-galaxie.org
Delivered-To: fhh@e-galaxie.org
X-Spam-Status: No
X-EGO-SI-From: fhh@e-galaxie.org
X-EGO-SI-SpamCheck: n'est pas un polluriel, SpamAssassin (score=-1.9,
  requis 6, autolearn=not spam, BAYES_00 -1.90,
  RCVD_IN_DNSWL_NONE -0.00)
X-EGO-SI: Found to be clean
X-EGO-SI-ID: 6BC656571.AD9A3
X-EGO-SI-Information: To informations about mails traitments contact
 your service informatique
Received: from olive.lsv.ens-cachan.fr (olive.lsv.fr [138.231.81.248])
  by mail.e-galaxie.org (Postfix) with ESMTP id 6BC656571
  for <fhh@e-galaxie.org>; Thu, 20 Oct 2016 14:36:40 +0200 (CEST)
Received: by olive.lsv.ens-cachan.fr (Postfix)
  id 29B9B4C01A9; Thu, 20 Oct 2016 14:36:40 +0200 (CEST)
Delivered-To: fhh@lsv.fr
Received: from ariane2.ens-cachan.fr (ariane2.ens-cachan.fr [138.231.176.54])
  by olive.lsv.ens-cachan.fr (Postfix) with ESMTP id 281AC4C0186
  for <fhh@lsv.fr>; Thu, 20 Oct 2016 14:36:40 +0200 (CEST)
Received: from localhost (localhost [127.0.0.1])
  by ariane2.ens-cachan.fr (Postfix) with ESMTP id 1B3FD142514
  for <fhh@lsv.fr>; Thu, 20 Oct 2016 14:36:39 +0200 (CEST)
X-Virus-Scanned: Debian amavisd-new at ens-cachan.fr
Received: from ariane2.ens-cachan.fr ([127.0.0.1])
  by localhost (ariane2.ens-cachan.fr [127.0.0.1]) (amavisd-new, port 10026)
  with ESMTP id 3XBACuUWF_q7 for <fhh@lsv.fr>;
  Thu, 20 Oct 2016 14:36:39 +0200 (CEST)
Received: from 08.dptinfo.ens-cachan.fr (08.dptinfo.ens-cachan.fr [138.231.36.108])
  (using TLSv1.2 with cipher ECDHE-RSA-AES256-GCM-SHA384 (256/256 bits))
  (No client certificate requested)
  by ariane2.ens-cachan.fr (Postfix) with ESMTPSA id BADA7142509
  for <fhh@lsv.fr>; Thu, 20 Oct 2016 14:36:39 +0200 (CEST)
Date: Thu, 20 Oct 2016 14:36:38 +0200
From: Francis HULIN-HUBARD <fhh@e-galaxie.org>
To: fhh@lsv.fr
Subject: Paf
Message-ID: <5808ba56./RUcSgsYhjKsCF+F%fhh@e-galaxie.org>
User-Agent: Heirloom mailx 12.5 6/20/10
MIME-Version: 1.0
Content-Type: text/plain; charset=us-ascii
Content-Transfer-Encoding: 7bit
X-Greylist: Sender IP whitelisted, not delayed by milter-greylist-4.3.7 
 (mail.e-galaxie.org [0.0.0.0]); Thu, 20 Oct 2016 14:36:40 +0200 (CEST)

1476966998
\end{verbatim}
\end{solution}

% }}}

% {{{ Codage des caractères en HTML

\section{Codage des caractères en HTML}

Les pages HTML disposent de plusieurs façons de coder des caractères :\\
\begin{itemize}
\item Les \emph{références d'entités} qui permettent de décrire des
  caractères non-ASCII par des caractères ASCII, et ça indépendemment
  du codage utilisé dans le reste du fichier. Ex. : \emph{à} s'écrit
  \texttt{\&agrave;} dans cette notation.\smallskip
\item Les \emph{références numériques} qui donnent le code d'un
  caractère Unicode, sous la forme \texttt{\&\#}\emph{nnn}\texttt{;}
  (décimal) ou \texttt{\&\#x}\emph{nnn}\texttt{;}.\smallskip
\item Le codage direct binaire dans l'un des formats évoqué dans
  le cours (UTF-8, ISO-8859-1, \ldots). Ceci nécéssite d'ajouter une
  balise de la forme :
\begin{quote}
\texttt{<meta http-equiv="Content-Type" content="text/html; charset=iso-8859-1" />}
\end{quote}
  dans le \texttt{header} du document HTML.\\
\end{itemize}

On va considérer les noms des communes européennes suivantes :
\begin{quote}
Crèvec\oe{}ur (France)\\
L'Ha\"y-les-Roses (France)\\
Krom\v{e}\v{r}\'i\v{z} (Tchéquie)\\
G\"od\"oll\H{o} (Hongrie)\\
S\"u\ss{}en (Allemagne)\\
Pr\ae{}st\o{} (Danemark)
\end{quote}

\begin{enumerate}
 \item Créez un document HTML qui affiche ces noms avec leurs références d'entités.\smallskip
 \item Cherchez les codes Unicode des caractères non-ASCII et utilisez-les pour recréer la liste avec des références numériques.\smallskip
 \item Créez un document HTML en UTF-8 directement avec votre éditeur de texte.
\end{enumerate}

% }}}

% {{{ Analyser une erreur de codage

\section{Analyser une erreur de codage}

Un grand classique que vous rencontrerez dans votre vie numérique est l'erreur de codage sur des pages web. Vos connaissances actuelles couplées à ce que vous avez vu en cours vous permettent de diagnostiquer \textbf{très précisément} le type de dysfonctionnement auquel vous êtes confrontés.\bigskip

Rendez-vous à l'url suivante : \url{http://www.lsv.fr/~fhh/tp05-1}\bigskip

Décrivez et détaillez ce qui se passe en utilisant les tables de caratères disponibles sur internet (\url{https://fr.wikipedia.org/wiki/Table_des_caract%C3%A8res_Unicode_(0000-0FFF)}). Proposez deux solutions permettant de corriger ce problème.

\begin{solution}

Nous sommes face à un problème d'encodage. Les caractères spéciaux s'affichent sur deux ou trois caractères au lieu d'un. Il semble que nous soyons face à une page UTF-8 interprétée en ISO8859-1.

\begin{itemize}
 \item é : à = 00c3 soit 11000011 ; © = 00a8 soit 10101000
 \item ç : 
 \item Ã\"{ } : 
\end{itemize}

\end{solution}

% }}}

% {{{ Codage des caractères dans le terminal

\section{Codage des caractères dans le terminal}

\subsection{Première observation en Bash}

Créer un fichier d'association Ville;Code postal contenant les entrées suivantes :

\begin{verbatim}
Paris;75000
Caen;14000
L'Haÿ-les-Roses;94240
Saint-Hilarion;78120
Charleville-Mézières;08000
Schœlcher;97233
\end{verbatim}

Écrire un code bash utilisant la commande <<printf>> pour présenter les entrées du fichier de manière alignées :

\begin{verbatim}
Paris                    75000
Charlevile-Mézières      08000
...
\end{verbatim}

Observez et expliquez le résultat.

\subsection{Seconde observation en C}

Adaptez le squelette suivant afin d'obtenir le même affichage que précédemment :

\begin{verbatim}
#include <stdio.h>

int main ()
{
        int i;
        char* villes[] = { "Saint-Étienne", "Charleville-Mézières", "Paris" };
        char* postal[] = { "42000", "08000", "75000" };

... Your code here ...

        return 0;
}
\end{verbatim}

Changez l'encodage des fichiers d'UTF-8 à ISO8859-1.

% }}}

% {{{ UTF-8

\section{UTF-8}                                                                                                                                                                                                                                                

\begin{enumerate}

\item Le code fourni est un programme C incomplet permettant de générer des
codes UTF-8 à partir d'un entier. Complétez la fonction \texttt{utf8}. Il
suffit de remplir les éléments de \texttt{buf}, en utilisant les bonnes
opérations binaires (\texttt{\&}, \texttt{$>>$}) sur la valeur cp.

\item Créez un programme qui fait l'inverse : décoder un flux d'octets codant
un texte en UTF-8 et émettre les codes Unicode qui composent le texte. Afficher
une erreur si jamais le flux contient un codage incorrect.
\end{enumerate}

% }}}

\end{document}
