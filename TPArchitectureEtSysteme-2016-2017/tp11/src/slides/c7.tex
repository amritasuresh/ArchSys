% Model checking course

\documentclass{beamer}
% \usepackage{folien}
% \usepackage{amsfonts}

\begin{document}

% --------------------------------------------------------------------------

\begin{frame}{Leader-election protocol}

\o The following protocol is due to Dolew, Klawe, Rodeh (1982).

\o The protocol consists of $n$ participants (where $n$ is a parameter).
   The participants are connected by a ring of unidirectional message channels.
   Communication is asynchronous, and the channels are reliable. Each
   participant has a unique ID (e.g., some random number).

\o \textbf{Goal:} The participants communicate to elect a ``leader''
   (i.e., some distinguished participant). The protocol shown here
   ensures low communication overhead ($\cal O(n\log n)$ messages;
   most naïve protocols have quadratic message overhead).

\end{frame}

% --------------------------------------------------------------------------

\begin{frame}{Leader-election protocol}

\o Participants are either active or inactive.
   Initially, all participants are \emph{active}.

\o The protocol proceeds in rounds. In each round, at least
   half of the participants will become inactive. (As a consequence,
   there are at most $\cal O(\log n)$ rounds.

\o In each round every active participant receives the numbers
   of the two nearest active participants (in incoming direction).
   A participant remains active only if the value of the nearest
   neighbour is the largest of the three. (In the following slides, the
   participant adopts this largest number as its own; this is optional.)

\o The last remaining active participant is declared the leader.

\end{frame}

% --------------------------------------------------------------------------

\begin{frame}{Leader Election: Example}

\o \centerline{\includegraphics[height=13cm]{lep1}}

\end{frame}

% --------------------------------------------------------------------------

\begin{frame}{Leader Election: First round}

\o \centerline{\includegraphics[height=13cm]{lep2}}

\end{frame}

% --------------------------------------------------------------------------

\begin{frame}{Leader Election: Result of the first round}

\o \centerline{\includegraphics[height=13cm]{lep3}}

\end{frame}

% --------------------------------------------------------------------------

\begin{frame}{Leader Election: Second round}

\o \centerline{\includegraphics[height=13cm]{lep4}}

\end{frame}

% --------------------------------------------------------------------------

\begin{frame}{Leader Election: Result of second round}

\o \centerline{\includegraphics[height=13cm]{lep7}}

\end{frame}

% ----------------------------------------------------------------------

\end{document}
