\documentclass[11pt]{article}
\usepackage[utf8]{inputenc}
\usepackage[T1]{fontenc}
\usepackage[francais]{babel}
\usepackage[francais]{layout}
\selectlanguage{french}

% NE PAS CHANGER !!
\ifx \public \undefined \def\public{etudiants} \fi
\usepackage[\public]{tps}

% Numéro du TP
\newcommand{\numtd}{09}
% Titre du TP
\newcommand{\titretd}{Petit hack entre amis}

\graphicspath{{imgs/}}

\begin{document}

\entete{\numtd}{\titretd}

\begin{introduction}
Ce TP revient sur les principes du linkage en C.
\end{introduction}

\section*{Édition de liens : linking}

On considère le fragment suivant d'un programme de C :
\begin{quote}
\texttt{x=y+z;}\\
\verb+printf("x=%d\n",x);+
\end{quote}

En analysant le code assembleur produit par le compilateur on trouve
que la première instruction va directement être réalisée par
des opérations sur les registres et la mémoire tandis que la deuxième
fait appel à une fonction externe.

\texttt{printf} est en effet réalisée par une fonction fournie par une
\emph{bibliothèque partagée}. Le modèle de compilation préféré
de Linux/Unix et d'autres systèmes est de stocker les programmes exécutables
dans les fichiers sous une forme incomplète, tout en spécifiant quelles
fonctions partagées seront nécéssaires à son exécution. Lors de
l'exécution, le programme sera chargé dans la mémoire, puis un
\emph{éditeur de liens} établit les connexions entre le programme
et les bibliothèques partagées : le <<\emph{linking}>> en anglais.

Normalement, le système sait trouver où trouver les bonnes librairies.
Le comportement de l'éditeur de liens peut être modifié par deux
variables :
\begin{itemize}
\item La variable \texttt{LD\_LIBRARY\_PATH} (qui peut être modifié
  sur la ligne de commande) donne une liste de répertoires où l'éditeur
  de liens pourra trouver des bibliothèques.
\item La variable \texttt{LD\_PRELOAD} définit quelques librairies dont
  l'utilisation sera prioritaire.
\end{itemize}

Dans cet exercise nous explorerons surtout des applications de l'utilisation de
\texttt{LD\_PRELOAD}.

\begin{enumerate}
\item On considère le programme \texttt{password1} (disponible sous forme
  binaire) qui demande un mot de passe. Trouvez le bon mot de passe.
\item Faites de même pour \texttt{password2}.
\item Comment éviter les failles de sécurité présentes dans ces
  deux programmes ?
\item Le programme \texttt{game} vous demande une numéro entre 1 et 100
  choisi aléatoirement lors de chaque exécution. Comment tricher pour
  gagner toujours au premier tour ?
\end{enumerate}

\begin{solution}
\begin{verbatim}
src
- game.c Source du jeux
- leone.c Illustration surcharge puts-hijack
- password1 Binaire simple plaintext
- password2 Linké attaque sur strcmp
- puts-hijack.c Bib hack pour leone


$ gcc -fPIC -c strcmp-hijack.c -o hack.o
$ gcc -shared hack.o -o hack.so
$ LD_PRELOAD=./hack.so ./password2 toto
\end{verbatim}
\end{solution}


\end{document}
