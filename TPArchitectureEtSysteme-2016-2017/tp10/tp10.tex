\documentclass[11pt]{article}
\usepackage[utf8]{inputenc}
\usepackage[T1]{fontenc}
\usepackage[francais]{babel}
\usepackage[francais]{layout}
\selectlanguage{french}

% NE PAS CHANGER !!
\ifx \public \undefined \def\public{etudiants} \fi
\usepackage[\public]{tps}

% Numéro du TP
\newcommand{\numtd}{10}
% Titre du TP
\newcommand{\titretd}{Les threads}

\graphicspath{{imgs/}}

\begin{document}

\entete{\numtd}{\titretd}

Page web du cours :                                                                                                                                                                                                                                            
\begin{center}
\url{http://www.lsv.ens-cachan.fr/~schwoon/enseignement/systemes/ws1617/}.
\end{center}

Sources du TP disponibles dans le répertoire <<\textasciitilde{}fhh/share/tp/tp10>> du département informatique.

Pensez \`a utiliser \texttt{man} pour d\'ecouvrir les d\'etails
des appels syst\`eme et des commandes dans le shell.

\section{S\'emaphores}

Rappelez-vous que les threads d'un m\^eme processus partagent leur m\'emoire
(sauf la pile) ; notamment ils partagent les variables statiques et
dynamiquement allou\'ees. On consid\`ere le programme suivant 
(\texttt{counter.c}):

\begin{verbatim}
#include <pthread.h>
#include <stdio.h>

#define MAX 100000
int ctr = 0;

void* count(void* data) {
  int i;
  for (i = 0; i < MAX; i++) ctr++;
}

int main () {
  pthread_t t1, t2;
  pthread_create(&t1, NULL, count, NULL);  // create first thread
  pthread_create(&t2, NULL, count, NULL);  // create second thread
  pthread_join(t1, NULL);  // wait for first thread
  pthread_join(t2, NULL);  // wait for second thread
  printf("Compteur: %d\n", ctr);
}
\end{verbatim}

\begin{enumerate}
\item[(a)] Quel est le r\'esultat attendu pour ce programme ?
\item[(b)] Qu'est-ce qui peut emp\^echer le r\'esultat attendu ?
\item[(c)] Quels r\'esultats sont (th\'eoriquement) possibles ?
\end{enumerate}

Les \emph{s\'emaphores} repr\'esentent une solution pour r\'esoudre le
probl\`eme des acc\`es concurrents aux variables. Un s\'emaphore est
une strucutre de donn\'ees utilis\'ee pour la synchronisation et pour
assurer un acc\`es coordonn\'e aux ressources partag\'es. En C,
les s\'emaphores sont d\'efinis par le fichier
\texttt{semaphore.h}. Les fonctions les plus importantes pour nos
objectifs seront les suivantes :
%
\begin{itemize}
\item
  \verb|int sem_init(sem_t *sem, int pshared, unsigned int value);|\\ 
  Initialise \verb|sem| avec la valeur donn\'ee ce qui donne le nombre
  de cr\'eneaux disponibles pour acc\'eder \`a une ressource au m\^eme temps.
  Pour nos objectifs, \verb|pshared| devrait \^etre \verb|NULL|.
\item \verb|int sem_wait(sem_t *sem);|\\ Si le s\'emaphore a la valeur 0
  cette fonction attend qu'un cr\'eneau devienne disponible. Sinon,
  elle d\'ecro\^it sa valeur imm\'ediatement.
\item \verb|int sem_post(sem_t *sem);|\\ Augmente la valeur de la
  s\'emaphore (un cr\'eneau devient disponible).
\end{itemize}
%
\'Etudiez les pages \verb|man| pour en savoir plus.

\begin{enumerate}
\item[(d)] Utilisez les s\'emaphores pour r\'eparer le programme \texttt{counter.c}.
\end{enumerate}

Enfin, consid\'erons le programme suivant (\texttt{order.c}) :
\begin{verbatim}
#include <pthread.h>
#include <semaphore.h>
#include <stdio.h>
#include <stdlib.h>

void* first(void* data)  { printf("First\n");  }
void* second(void* data) { printf("Second\n"); }
void* third(void* data)  { printf("Third\n");  }

int main () {
  pthread_t t1, t2, t3;
  
  pthread_create(&t3, NULL, third, NULL);
  pthread_create(&t2, NULL, second, NULL);
  pthread_create(&t1, NULL, first, NULL);

  // wait for all threads
  pthread_join(t1, NULL);  pthread_join(t2, NULL); pthread_join(t3, NULL);
}
\end{verbatim}
Modifiez le programme pour assurer que le programme donne toujours la
 sortie suivant :
\begin{verbatim}
  First
  Second
  Third
\end{verbatim}

\begin{enumerate}
\item[(e)]
 \'Ecrivez un programme en C avec deux threads. Le premier thread
 affiche les entiers pairs jusqu\`a 100, le deuxi\`eme les impairs.
 Utilisez des s\'emaphores pour assurer le bon ordre des entiers.
\end{enumerate}

\section{Producer/consumer}

Un probl\`eme typique dans la programmation concurrente est celui du
\emph{producteur et consommateur} : un processus produit des donn\'ees
que l'autre consomme. On consid\`ere le programme 
\texttt{procon.c}. Ici, le producteur obtient des caract\`eres dans un
fichier et les envoie vers un espace partag\'e avec le processus consommateur
qui les affiche sur l'\'ecran. Cependant, dans la version actuelle,
aucune synchronisation existe entre les deux.

Modifiez le programme tel qu'il affiche le contenu d'un fichier correctement.

\section{Mandelbrot}

Soit $c$ un nombre complexe. On consid\`ere la s\'erie
$z_0:=0$ et $z_{n+1}:=z_n^2+c$ pour $n\ge0$. L'\emph{ensemble de Mandelbrot}
est d\'efini comme l'ensemble des valeurs $c$ telles que la s\'erie des $z_n$
est born\'ee. On sait que cela est le cas si $z_n$ ne sort jamais d'un
cercle de radius 2 autour de 0. Si jamais la s\'erie sort de ce cercle,
soit $m(c)$ le plus petit indice $n$ tel que c'est le cas.

Une application populaire pour $m(c)$ est de cr\'eer
de jolies images ; on associe l'\'ecran avec un rectangle et chaque pixel
avec la valeur $c$ qui y correspond ; le pixel est ensuite peint avec
une couleur correspondant \`a $m(c)$.

La page web du cours contient un tel programme. Votre t\^ache consiste
en l'accelerant en lan\c{c}ant plusieurs threads en parall\`ele.
Les machines de la salle 411 sont \'equip\'ees de plusieurs c\oe{}urs
(utilisez \texttt{cat /proc/cpuinfo} pour trouver plus d'information).
Chaque thread va donc travailler sur une partie diff\'erent de l'image.

Note: Compilez le programme avec \texttt{make}.

Note (2): Cet exercice ne n\'ecessite pas de s\'emaphores.

Pages man utiles : pthreads(7), pthread\_create(3), pthread\_join(3).

\end{document}
