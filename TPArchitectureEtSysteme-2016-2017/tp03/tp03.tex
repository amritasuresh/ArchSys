\documentclass[11pt]{article}
\usepackage[utf8]{inputenc}
\usepackage[T1]{fontenc}
\usepackage[francais]{babel}
\usepackage[francais]{layout}
\selectlanguage{french}

% NE PAS CHANGER !!
\ifx \public \undefined \def\public{etudiants} \fi
\usepackage[\public]{tps}

% Numéro du TP
\newcommand{\numtd}{03}
% Titre du TP
\newcommand{\titretd}{Circuits logiques}

\graphicspath{{imgs/}}

\begin{document}

\entete{\numtd}{\titretd}

\begin{introduction}
 Page web du cours :
 \begin{center}
  \url{http://www.lsv.ens-cachan.fr/~schwoon/enseignement/systemes/ws1617/}
 \end{center}

 Utilisez la page suivante pour construire vos circuits :
 \begin{center}
  \url{http://www.neuroproductions.be/logic-lab/}
 \end{center}
\end{introduction}

\section{Codeur}

Un \emph{codeur} est l'opposé du \emph{décodeur} discuté en cours ;
il prend $2^k$ signaux en entrée ($x_0\cdots x_{2^k-1}$) et fournit un
vecteur de $k$ sorties $y_{k-1}\cdots y_0$ représentant une valeur $y$.
Si $k$ est fixe, on parle d'un $k$-codeur.
Dans un codeur, on suppose qu'exactement une des sorties a la valeur 1,
disons $x_i$. Dans ce cas, la valeur binaire des sorties est censé être $i$.
Le comportement pour d'autres cas n'est pas spécifié.

\begin{enumerate}
 \item Construisez un 1-codeur. (trivial)
 \item Construisez un 2-codeur.
 \item Comment géneraliser la construction pour un $k$ quelconque ?
	Quelle est la taille et la profondeur de votre construction
	par rapport à $k$ ?
 \item Construisez un 2-codeur avec l'application web mentionnée ci-dessus.
\end{enumerate}

\begin{solution}
 \begin{enumerate}
  \item
 \begin{tabular}{|l|c|r|}
  \hline
  $x_1$ & $x_0$ & $y_0$ \\
  \hline
  0 & 1 & 0 \\
  1 & 0 & 1 \\
  \hline
 \end{tabular}
 \end{enumerate}
\end{solution}

\section{Codeur de priorité}

Un \emph{codeur de priorité} (CP) est comme un codeur, mais il gère le
cas où plusieurs entrées ont la valeur 1. Dans ce cas, $y$ prend
la valeur du plus grand indice $i$ tel que $x_i=1$. Une autre sortie $z$
indique si au moins un des $x_i$ était $1$. Si $z=0$, la valeur de $y$
n'est pas spécifiée.

\begin{enumerate}
 \item Selon vous, à quoi peut servir un tel circuit ?
 \item Construisez un CP pour 2, puis 4 signaux avec l'application web.
 \item Étant donné deux $k$-CP, décrivez comment
   construire efficacement un $(k+1)$-CP.
 \item Décrivez comment construire un $2k$-CP à partir des $k$-CP
   (et d'autres circuits discutés en dans le cours).
\end{enumerate}

\iffalse
\section{Verrou RS}
\begin{enumerate}
 \item Construisez un verrou RS, mais avec des NON-ET au lieu des NON-OU.
  Quel sera le comportement des sorties $Q,\bar Q$ en fonction des entrées
  $R,S,Q, \bar Q$ ?
 \item Comparez le comportement d'un verrou construit avec des portes au
  verrou directement disponible dans l'appli.
\end{enumerate}

\section{Verrou T et compteur}
\begin{enumerate}
 \item Expérimentez avec le verrou T disponsible dans l'appli. Qu'est-ce qu'il
  fait ? (Essayez-le avec un interrupteur et un impulseur en entrée.)
 \item Chargez l'exemple qui simule un compteur binaire (en bas de la page).
  Modifiez-le pour qu'il augmente le compteur au lieu de le décroître.
\end{enumerate}
\fi

\end{document}
