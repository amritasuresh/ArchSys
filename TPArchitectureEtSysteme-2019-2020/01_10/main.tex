\documentclass{exam}
\usepackage{../main}

\if\answers0\noprintanswers\fi
\if\answers1\printanswers\fi

\title{DM et babouins}
\author{}
\date{}

\begin{document}
\maketitle
\section{Babouins et gu\'epards}
Mod\'elisons un syst\`eme proies-pr\'edateurs par des processus.  Un gu\'epard
attaque un babouin via l'envoi d'un signal. La reproduction se fait par un
fork.

On note \( B \) le nombre de babouins et \( G \) le nombre de gu\'epards.
Un gu\'epard attaque un babouin avec une probabilit\'e \( a \). Un
babouin se reproduit avec une probabilit\'e \( r_b \) et un gu\'epard
avec une probabilit\'e \( r_g \).

Lorsqu'un gu\'epard lance une attaque, il choisit une cible. Pour cela, on
allouera un espace m\'emoire partag\'e via \texttt{shmem} contenant un tableau
avec tous les \textit{pids} des babouins. Pour se reproduire un individue d'une
esp\`ece \'emet une invitation avec une proababilit\'e \( r_b\) \`a un autre
individu, qui accepte avec une probabilit\'e \( r_b \).

Chaque individu attend un temps \( \tau \) entre chaque action.
\begin{questions}
  \question\
  Quel signal utiliser pour tuer les babouins? Quel est le masque des
  signaux \`a traiter?
  \question\
  Comment pourrions nous d\'ecider quel babouin tuer, en supposant que
  l'on n'utilise pas de m\'emoire partag\'ee?
  \question\
  Impl\'ementer et trouver des valeurs de probabilit\'es menant \`a un
  \'equilibre. Ne pas oublier l'existence de \texttt{getpid}.
  \question\
  Les deux \'esp\`eces subissant une \'evolution comportementale, les
  individus demandent l'avis de leur partenaire avant de se reproduire.
  Si le partenaire est d'accord (avec la m\^eme probabilit\'e), alors la
  reproduction se fait. Peut-on encore utiliser \texttt{sa\_sigaction}?
  (pour simplifier, un individu continue de vivre normalement tant que le
  partenaire n'a pas accept\'e)
  \question\
  Si un babouin perd un de ses enfants, il peut tuer un
  gu\'epard avec une probabilit\'e \( e \). Quel signal utiliser? Quels
  masques faut-il changer? Impl\'ementer.
  \question\
  Les babouins se nourissent de la flore environnante. S'ils sont trop
  nombreux, le manque de ressources devient dangereux. Comment
  impl\'ementer ce m\'ecanisme?
  \begin{solution}
    \inputminted{C}{baboons/main.c}
  \end{solution}
\end{questions}

\end{document}
