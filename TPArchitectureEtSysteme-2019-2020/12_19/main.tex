\documentclass{exam}
\usepackage{../main}

\if\answers0\noprintanswers\fi
\if\answers1\printanswers\fi

\title{Réseau}
\date{}

\begin{document}
\maketitle
\begin{questions}
\titledquestion{Élection d'un meneur}
\begin{parts}
  \part{} La version réseau du programme, <<ring-net.c>> instancie un
  seul n\oe{}ud par exécution. Vous pouvez recopier votre fonction
  <<protocole>> depuis <<ring-pipe.c>>.
  Le programme prend 3 arguments~: le port d'écoute du n\oe{}ud, le nom du nœud voisin (nom de
  machine), le port de connexion au nœud voisin. Le programme crée un
  serveur dans un thread et attend la connexion d'un voisin. En
  parallèle, dans un autre thread, il tente une connexion sur son
  voisin (nom de machine et port passés en argument). Écrivez un
  script mettant en œuvre l'exécution de votre code sur 5 machines du
  département. Commencez par tester l'exécution de votre code
  localement (machine localhost en utilisant différents
  ports). Étendre progressivement la taille de l'anneau.
  \begin{solution}
    \inputminted{C}{../12_13/correc/ring/ring-net.c}
  \end{solution}
  \part{} À partir des ordinateurs de la salle informatique
  (connectez vous en SSH si vous n'y êtes pas),
  créez un anneau avec vos voisins et l'étendre à toute la
  salle de TP.
\end{parts}

\titledquestion{Discussions}
\begin{parts}
  \part{}
On se propose de coder une application pour discuter entre nous sur le terminal.
Il faut donc un programme permettant à la fois de recevoir les messages et de
les envoyer. Pour cela, on crééra ou bien deux threads, ou bien deux processus.
L'un d'eux agira en tant que serveur, et l'autre en tant que client. On pourra
se servir des fonctions disponibles dans
\url{http://www.lsv.fr/~hondet/resources/archos/packets.c} et
\url{http://www.lsv.fr/~hondet/resources/archos/packets.h}.
\begin{solution}
  \inputminted{C}{chat/client.c}
\end{solution}

  \part{}
  Comment faire pour recevoir des messages de plusieurs personnes? Envoyer des
  messages à plusieurs personnes? Implémenter.
\begin{solution}
  Lancer un fil par hote distant.
\end{solution}
\end{parts}
\end{questions}
\end{document}
