\documentclass{exam}
\usepackage{fontspec}
\usepackage{polyglossia}
\usepackage{hyperref}
\usepackage{listings}
\lstset{
  language=C,
  basicstyle=\ttfamily
}
\usepackage{dirtree}
\setmainlanguage{french}

\title{Multi Core Unix wc}
\author{}
\date{}
\begin{document}
\maketitle

On se propose de réimplémenter l'utilitaire unix \texttt{wc} en multi-c\oe{}urs.
\texttt{whatis} nous dit
\begin{verbatim}
wc (1)       - print newline, word, and byte counts for each file
\end{verbatim}

En supposant que l'exécutable se nomme \texttt{mwc}, la commande
\begin{verbatim}
mwc -w FILE
\end{verbatim}
doit renvoyer le nombre de mots du fichier \verb|FILE|. Pour compter le nombre de 
lignes, on utilisera \verb|mwc -l FILE|, et pour compter le nombre de caractères,
\verb|mwc -c FILE|. Le cas sans option doit agir de la même manière qu'avec l'option \verb|-w|. 
On ne traite qu'une seule option à la fois, le programme n'utilise pas l'entrée standard.

Le comportement (et la sortie) de votre \texttt{mwc} doit donc être
rigoureusement le même que celui du \texttt{wc} Unix à option
identique (sauf le cas sans option).

Le principe est de diviser le fichiers et d'attribuer un morceau de fichier par
fil d'exécution.

\url{http://www.lsv.fr/~hondet/resources/archos/mwc.tar.gz} contient un
Makefile et une batterie de tests. Pour lancer les tests, \texttt{make
  tests}. Pour générer des fichiers de taille arbitraire (pour tester
les performances),
\begin{verbatim}
< /dev/urandom tr -dc '\n\t [:alnum:]' | head -cN > FILE
\end{verbatim}
où \texttt{N} est la taille du fichier et \texttt{FILE} est le fichier
de sortie.

\paragraph{Remarques}
\begin{itemize}
  \item On utilisera des headers (fichiers en \texttt{.h}).
  \item On privilégiera les appels système\footnote{En règle générale, il vaut
    mieux privilégier les appels de librairies, plus génériques et portables.}.
    On utilisera donc \texttt{open} plutôt que \texttt{fopen} (\texttt{printf}
    est autorisé).
  \item Il faut toujours s'attendre à un échec d'un appel système. Afficher
    l'erreur avec \texttt{perror} sera un traitement suffisant.
  \item Le manuel Unix est très utile lorsqu'on programme avec les librairies
    Unix, la section 2 est dédiée aux appels système, la section 3 concerne les
    appels de librairie (les pages relatives à \texttt{pthread} sont en section
    3).
  \item On fera attention à ne pas avoir de fuites mémoire. Pour s'en assurer,
    on peut utiliser l'outil \texttt{valgrind}:
\begin{verbatim}
valgrind ./mwc file.plain
\end{verbatim}
\end{itemize}

\paragraph{Challenge}
L'archive contient le binaire de ma solution (qui utilise 4
c\oe{}urs). Comparez, et faites mieux.

\paragraph{Modalités de rendu}
Vous rendrez une archive tar compressée nommée \texttt{Nom\_Prénom.tar.gz}
dont l'architecture sera:
\dirtree{%
  .1 Nom\_Prénom.
  .2 readme.pdf.
  .2 Makefile.
  .2 main.c.
  .2 main.h.
  .2 additional files.c.
  .2 additional files.h.
  .2 tests.
  .3 \dots.
}
où \texttt{readme.pdf} contient des indications si vous jugez cela nécessaire.
On notera que les fichiers sont présents directement à la racine de l'archive.
Le \texttt{Makefile} doit avoir deux cibles
\begin{itemize}
  \item \texttt{bin} qui permet de générer le binaire et
  \item \texttt{tests} permettant de tester l'exécutable créé.
\end{itemize}
Pour rappel, une cible s'écrit
\begin{verbatim}
cible: dépendances
	recette
\end{verbatim}
o\`u la recette est la commande déclenchée par la cible. Les dépendances peuvent
être vides.

La propreté du code influencera la note. Vous pouvez utiliser des
embellificateurs de code comme
\begin{itemize}
  \item \url{http://astyle.sourceforge.net/} ou
  \item \url{http://uncrustify.sourceforge.net/} ou encore
  \item \url{https://clang.llvm.org/docs/ClangFormat.html}.
\end{itemize}

Vous trouverez une archive
\url{http://www.lsv.fr/~hondet/resources/archos/mwc.tar.gz} pour
démarrer. Elle contient l'arborescence demandée.
Pour toute question, n'hésitez pas \`a contacter la hotline:
\texttt{gabriel.hondet@lsv.fr}.
\end{document}

%%% Local Variables:
%%% mode: latex
%%% TeX-master: t
%%% End:
