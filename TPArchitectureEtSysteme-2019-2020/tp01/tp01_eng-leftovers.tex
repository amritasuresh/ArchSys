\documentclass[11pt]{article}
\usepackage[utf8]{inputenc}
\usepackage[T1]{fontenc}
\usepackage[francais]{babel}
\usepackage[francais]{layout}
\usepackage{hyperref}
\selectlanguage{french}

% NE PAS CHANGER !!
% \ifx \public \undefined \def\public{etudiants} \fi
\ifx \public \undefined \def\public{enseignants} \fi
\usepackage[\public]{tps}

% Numéro du TP
\newcommand{\numtd}{01}
% Titre du TP
\newcommand{\titretd}{The command line in Linux}



\begin{document}

\entete{\numtd}{\titretd}




\section{Install a program}

You are not allowed to install packages (files for deploying programs on Linux) on the computers in Room 411, however, you can compile and install a program or a library that you would like to use in your home directory. In this exercise, we will compile a recent version of OpenSSH in our "home" for use in the next TP.

\begin{itemize}
	\item Download OpenSSH sources to an src directory at the root of your account via the "wget" command;
	\begin{solution}
		
		mkdir \textasciitilde{}/src \&\& cd \textasciitilde{}/src
		
		wget http://ftp.fr.openbsd.org/pub/OpenBSD/OpenSSH/portable/openssh-7.3p1.tar.gz
	\end{solution}
	\item Extraire les sources à l'aide de la commande <<tar>> ;
	\begin{solution}
		tar xf openssh-7.3p1.tar.gz
	\end{solution}
	\item Configurer les sources pour installer le programme dans \textasciitilde{}/Applications
	\begin{solution}
		
		mkdir \textasciitilde{}/Applications
		
		cd openssh-7.3p1
		
		./configure $--$prefix=\$HOME/Applications
	\end{solution}
	\item Compilez les sources en utilisant la commande <<make>>
	\begin{solution}
		make -j8
	\end{solution}
	\item Installez le programme, que se passe-t-il ? Corrigez si besoin.
	\begin{solution}
		
		make install
		
		...
		
		mkdir: cannot create directory ‘/var/empty’: Permission denied
		
		Le programme tente de créer un répertoire dans /var ce qui est interdit.
		
		La lecture de <<./configure $--$help>> nous apprend que <<$--$with-privsep-path=...>> permet de définir le positionnement du répertoire <<empty>>.
		
		Relançons le configure en ajoutant $--$with-privsep-path=\$PREFIX
		
		make clean
		
		./configure $--$prefix=\$HOME/Applications $--$with-privsep-path=\$HOME/Applications
		
		make -j8
		
		make install
		
		...
		
		make: [check-config] Error 255 (ignored)
		
		Une erreur est signalée pour le changement de droit sur certain fichiers. L'erreur est affichée comme ignorée.
		
	\end{solution}
	\item Vérifiez la version de SSH et commentez le résultat.
	\begin{solution}
		
		ssh -V
		
		OpenSSH\_6.6.1p1 Ubuntu-2ubuntu2.8, OpenSSL 1.0.1f 6 Jan 2014
		
		Notre version n'est pas utilisée car elle n'est pas dans le chemin de recherche des binaires.
		
	\end{solution}
	
	\item Ajouter le répertoire \textasciitilde{}/Applications/bin au PATH
	\begin{solution}
		
		PATH=\$HOME/Applications/bin:\$PATH
	\end{solution}
	
	\item Vérifiez la version de SSH et commentez le résultat.
	\begin{solution}
		
		ssh -V
		
		OpenSSH\_7.3p1, OpenSSL 1.0.1f 6 Jan 2014
		
		Notre version est prise en compte.
	\end{solution}
	\item Rendrez la modification pérenne
	\begin{solution}
		\begin{lstlisting}
		cat >> ~/.bashrc
		PATH=\$HOME/Applications/bin:\$PATH
		^d
		\end{lstlisting}
	\end{solution}
\end{itemize}



\section{Gallerie photo}

L'objectif de cette partie est de créer un script générant une gallerie photo html.

Dans un répertoire se trouve des photos aux formats PNG et JPG. Ce répertoire peut contenir des sous répertoires, correspondant aux collections de photos.

Nous souhaitons un script bash prenant en argument le répertoire contenant les photos et générant une page html contenant la liste des collections et une icône associée, puis une icône de chaque image classée par date de prise de la photo. Losque l'on clique sur une icone de photo, la photo apparaît en 1024x768 (en conservant le ratio longeur largeur). Les images sont pivotées en fonction de la prise de vue. Dans le répertoire contenant les photos, ou les collections, un fichier "comment.txt" permet de définir des commentaires au format "nom\_photo~:~Commentaire sur la photo" qui sera affiché sous la photo lors de l'affichage plein écran.

Toutes les évolutions imaginées sont les bienvenues.

Pour extraire la position des photos et les dates de prises de vues, vous utilisez le programme <<exiv2>>.

Les icônes des images appelées <<thumbnail>> seront générées via la commande <<convert>>. Les icônes déjà générées ne seront pas regénérées (utilisez le md5sum des fichiers comme nom d'icône par exemple).

\begin{solution}
	\begin{itemize}
		\item téléchargement de exiv2 http://www.exiv2.org/download.html
		\item extraire les sources
		\item créer le répertoire de vos applications <<mkdir \textasciitilde{}/Applications>>
		\item configurer exiv2 depuis le répertoire des sources : ./configure $--$prefix=\$HOME/Application
		\item compilez le programme : make
		\item installer le programme dans votre répertoire : make install
		\item adaptez le PATH en cas de besoin : PATH=\$PATH:\$HOME/Applications/bin/
		\item testez la commande : exiv2
		\item création d'une image thumbnail : convert file1 -thumbnail 200x200 f.md5sum.jpg
	\end{itemize}
\end{solution}







\section{Scripts}

A script is a set of instructions written in a file that will be executed sequentially.

\textbf{In the rest of this TP we will use the bash command interpreter.
}

All instruction blocks in a script can be tested directly in the shell.

\begin{lstlisting}
cat > myscript
cd /tmp
ls
cd
^d
\end{lstlisting}

The script runs from the command line:

\begin{lstlisting}
bash myscript
\end{lstlisting}

In this example we specify the shell to use ("bash") and it passes in the arguments the file containing the instructions.

We could also specify, directly in the script, what command interpreter to use by specifying it in the header of the file

\begin{lstlisting}
cat > myscript
#!/usr/bin/bash

cd /tmp
ls
cd
^d
\end{lstlisting}

This first line is called \textbf{"shebang"}.

The "whereis" command is used to locate the shell you want to use:

\begin{lstlisting}
whereis bash
\end{lstlisting}

the script can then be invoked like any other program.

\begin{lstlisting}
./myscript
\end{lstlisting}

It does not work, why? Correct it.

\begin{solution}
	The script does not have execute rights.
	
	chmod +x myscript
\end{solution}

\subsection{Control structure}

"man test" gives the list of tests available in bash.

\subsubsection{if .. then ... else}

Syntax:

\begin{verbatim}
if commande(s) ; then
commande(s)
[ elif commande(s) ; then
commande(s) ] ...
[ else commande(s) ]
fi
\end{verbatim}

Using "if" to test the existence of the directory "/var/logs", if it doesn't exists check whether the directory "/var/lOg" exists.  Display a message in case both directories are found or not or one of them is found.

\begin{solution}
	\begin{verbatim}
	if [ -d /var/logs ] ; then echo logs found ; \
	elif [ -d /var/lOg ] ; then echo log found ; \
	else echo none of them exists ; \
	fi
	\end{verbatim}
\end{solution}

\textbf{NB :} In the case of long command line, you can use "\textbackslash{}" to go down a line and continue your command.

Write the message "Alert" if you have the right to write in "/etc".

\begin{solution}
	\begin{lstlisting}
	[ -w /etc ] && echo Alerte
	\end{lstlisting}
\end{solution}

\subsubsection{for ...  do ... done}

Syntax :

\begin{verbatim}
for var in list ; do
commande(s)
done
\end{verbatim}
Write a "for" which displays the letters "a b c d", one per line.

\begin{solution}
	\begin{verbatim}
	for i in a b c d ;  do echo $i ; done
	\end{verbatim}
	Illustrate global expressions and type lists {1..10}
\end{solution}

\subsubsection{while ... do ... done}

Syntax :

\begin{verbatim}
while commande(s) ; do
commande(s)
done
\end{verbatim}

What does the following code do?

\begin{lstlisting}
ls | while read a ; do echo $a ; done
\end{lstlisting}

\begin{solution}
	Displays one per line the files in the current directory.
\end{solution}

\subsection{Network scanner}
Write a loop testing your network environment. You will use "ping" to attempt to call the 254 machines in your IP range and will only display available IP addresses (that have responded to ping).

Ecrire une boucle testant votre environnement réseau. Vous utiliserez <<ping>> pour tenter d'appeler les 254 machines de votre plage IP et n'afficherez que les adresses IP disponibles (ayant répondu au ping).

\begin{solution}
	\begin{verbatim}
	for i in {1..254} ; do
	ping -c1 -w1 138.231.81.$i > /dev/null && echo "138.231.81.$i"
	done
	\end{verbatim}
\end{solution}

\subsection{Trash script}

Write a small script that takes a set of files as a parameter and moves them to the trash (the ".trash" directory at the root of your user directory).
Display an error message if a file does not exist.
First, find the special variable that represents all the parameters of the command line (see man bash).

The little script should be between 10 to 15 lines.

\begin{solution}
	\begin{verbatim}
	#!/bin/bash
	
	[ -d ~/.trash ] || mkdir ~/.trash
	for i in $* ; do
	if [ -e "$i" ] ; then
	mv "$i"  ~/.trash
	else
	echo "$i not found"
	fi
	done
	\end{verbatim}
\end{solution}
\end{document}
