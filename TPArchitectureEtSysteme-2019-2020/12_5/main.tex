\documentclass{exam}
\usepackage{../main}
\usepackage{amsmath}

\if\answers0\noprintanswers\fi
\if\answers1\printanswers\fi

\title{Fils et semaphores}
\begin{document}
\maketitle

\begin{questions}
\titledquestion{Spinlock v. Mutex}
Quelle est la différence entre un \textit{spinlock} et un
\textit{mutex}? Pour mettre en évidence cette différence, téléchargez
\url{http://www.lsv.fr/~hondet/resources/archos/spinvmut.c} et complétez le
de manière a ce que le fil \texttt{stupid} s'accapare le spinlock et
que \texttt{wants} soit forcé d'attendre la libération de la
ressource (exclusion mutuelle).

La commande \verb|time cmd| mesure le temps d'occupation CPU en mode kernel
(\textit{sys}), le  temps d'occupation CPU en mode utilisateur (\textit{user})
et le temps réel (\textit{real}),
écoulé, présenté sous la forme
\begin{verbatim}
real    0m0.000s
user    0m0.000s
sys     0m0.000s
\end{verbatim}
si vous utilisez la command \textit{bash}. Il existe aussi un binaire
\verb|time|, qui présente les résultats de cette manière,
\begin{verbatim}
0.00user 0.00system 0:00.00elapsed ?%CPU (0avgtext+0avgdata 1808maxresident)k
\end{verbatim}
où le temps réel est noté \textit{elapsed}.

Que donnera un appel à \verb|time spinvmut| concernant le temps réel et le temps
CPU? Modifiez le code pour utiliser des mutex à la place. Comment devrait
évoluer le temps réel et le temps CPU, en lançant \verb|time spinvmut|?
Vérifiez.

\begin{solution}
  Version avec spinlocks,
  \inputminted{C}{corrected/spinvmut.c}
  ce qui donne, par un appel à \verb|time|,
  \begin{verbatim}
  At last!

  real    0m4.002s
  user    0m2.998s
  sys     0m0.004s
  \end{verbatim}
  Version avec mutex,
  \inputminted{C}{corrected/spinvmut_mut.c}
  ce qui donne, par un appel à \verb|time|,
  \begin{verbatim}
  At last!

  real    0m4.002s
  user    0m0.003s
  sys     0m0.004s
  \end{verbatim}
  où l'on voit que, bien que les temps réels soient égaux, on a un différence
  entre les temps CPU: le \textit{spinlock}, car il réalise une attente active,
  occupe le processeur, on a donc un temps CPU élevé. Le \textit{mutex}, réalisant
  une attente passive, libère les ressources, et prend donc peu de temps CPU.
\end{solution}

\titledquestion{La fonction d'Hénon, partie II}
On va (re) calculer l'orbite d'un système dynamique de dimension 2. La fonction
d'Hénon est définie par le système
\begin{equation}
  \label{eq:1}
  H_{a,b} = \left\{
    \begin{aligned}
      x_{n + 1} &= a - by_{n} - x_{n}^{2} \\
      y_{n + 1} &= x_{n}.
    \end{aligned}
    \right.
\end{equation}
On utilisera un thread pour calculer la suite \( (x_n)_n \) et un
autre thread pour la suite \( (y_n)_n \).

Proposez un moyen de synchronisation permettant d'assurer
l'entrelacement des calculs de \( x_n \) et de \( y_n \). Mettez le en
\oe{}uvre.

On peut ensuite créer un modèle dit producteur consommateur:
\begin{itemize}
\item un thread calcule l'orbite et stocke les données dans un buffer
  tournant,
  \item l'autre thread lit les données du buffer et les écrit dans un fichier.
\end{itemize}
Le fichier \texttt{henon.dat} sera sous la forme
\texttt{0.3415 1.2451} où le premier nombre est $x_{n}$ et le deuxième $y_{n}$.

On pourra tracer la fonction avec la commande \texttt{gnuplot henon.p} après
avoir téléchargé le script
\url{http://www.lsv.fr/~hondet/resources/archos/henon.p}. Cela produira un
fichier \texttt{henon.png}.

Pour \( a = 1.4 \) et \( b = -0.3 \), la fonction est chaotique (au sens de
Devaney), c'est-à-dire,
\begin{itemize}
  \item la fonction est sensiblement dépendante des conditions initiales (effet
    papillon, imprédictabilité);
  \item la fonction est topologiquement transitive (indécomposabilité);
  \item les points périodiques sont denses dans le domaine de définition de
    $H_{a, b}$ (régularité).
\end{itemize}
et possède un attracteur étrange.

Pour plus d'informations sur les systèmes dynamiques, voir \textit{An
  introduction to Chaotic Dynamical Systems}, Robert~L.~Devaney, Westview.

\begin{solution}
  \paragraph{Simple}
  \inputminted{C}{corrected/henon/henon_simple.c}
  \paragraph{Producteur-consommateur}
  \inputminted{C}{corrected/henon/henon_pc.c}
\end{solution}

\titledquestion{Mandelbrot}
Soitcun nombre complexe. On considère la série \(z_0 = 0\) et 
\(z_{n+1} = z_n^2 + c \) pour \(n \geq 0\).
L’ensemble de Mandelbrot est défini comme l’ensemble des valeurs \(c \)
telles que la série des \(z_n\) est bornée. On sait que cela est le cas si
\(z_n\) ne sort jamais d’un cercle de rayon 2 autour de 0.
Si jamais la série sort de ce cercle, soit \(m(c)\) le plus petit indice 
\(n \) tel que c’est le cas. Une application populaire pour \(m(c)\) est de
créer de jolies images ; on associe l’écran avec un rectangle et chaque pixel
avec la valeur \(c\) qui y correspond ; le pixel est ensuite peint avec une couleur
correspondant à \(m(c)\). La page web du cours contient un tel programme. Votre tâche
consiste à l’accélérer en lançant plusieurs threads en parallèle.
On peut utiliser \verb|cat /proc/cpuinfo| pour voir combien de c\oe urs une machine a.
Chaque thread va donc travailler sur une partie différente de l’image.

\titledquestion{Sémaphores inter processus (facultatif)}
Les semaphores, bien qu'introduits avec les threads, permettent
également la synchronisation inter processus. Pour ce faire, l'API
System V fournit des procédures \textit{IPC} pour \textit{Inter
Process Communication}. Par exemple, on peut obtenir des sémaphores
via \texttt{semget}. Cette commande utilise des clefs IPC permettant
d'identifier de manière unique un ensemble de sémaphores.

Ces clefs peuvent être générées par la fonction \texttt{ftok}.

Pour expérimenter un peu, écrivez un programme créant une clef et deux
sémaphores avec cette clef. Lancez le programme et exécutez la
commande \texttt{ipcs -s}. Que constate-t-on? On en déduira les
précautions à prendre lorsque l'on utilise des clefs. La commande
\texttt{ipcrm} existe.

On peut maintenant implémenter la fonction d'Hénon multi processus
avec des sémaphores. L'utilisation de sémaphores partagés reste assez
fastidieuse.

\begin{solution}
  On donne un exemple d'utilisation des mémoires partagées, en deux programmes.
  L'écrivain crée une mémoire et écrit dedans. Le lecteur va lire la mémoire.
  \paragraph{Écrivain}
  \inputminted{C}{corrected/shmem_writer.c}
  \paragraph{Lecteur}
  \inputminted{C}{corrected/shmem_reader.c}
\end{solution}

\end{questions}
\end{document}
