\documentclass{exam}
\usepackage[]{tps}

%\if\answers0\noprintanswers\fi
%\if\answers1\printanswers\fi

\title{Processus \& tubes}
\author{Gabriel Hondet\\\texttt{gabriel.hondet@lsv.fr}}

\begin{document}
\maketitle

\begin{questions}
  \titledquestion{Fichiers louches}
  Téléchargez le fichier
  \url{http://www.lsv.fr/~hondet/resources/archos/touch_weird.sh.x} et exécutez
  le.
  Trois fichiers (vides) devraient être apparus dans votre répertoire courant.
  Supprimez-les en utilisant \emph{uniquement} la ligne de commande.\\
  \emph{Indication:} comment sont identifiés les fichiers de manière générale?

  \begin{solution}
    Utiliser les inodes: \verb|ls -i| et \verb|find . -inum ... -exec rm {} \;|
  \end{solution}

\titledquestion{Programme erroné}
Considérons
\url{http://www.lsv.fr/~hondet/resources/archos/closed_pipe.c}.
Le fils est censé imprimer les caractères que lui envoie le père: expliquez le
dysfonctionnement et corrigez le programme.

\begin{solution}
  The read end \texttt{p[0]} of the pipe is closed before forking, hence the
  child cannot read from it. Moving \verb|close(p[0])| below
  \verb|if (fork()) { | solves the problem.
\end{solution}

\titledquestion{Tubes et remplacement de code}
Écrivez un programme (en C) qui télécharge l'archive
\url{http://www.lsv.fr/~hondet/resources/archos/shell-bootstrap.tar.gz},
et la décompresse sans créer de fichier temporaire. Dit autrement, on veut
coder la commande \verb+curl <url> | tar xz+ en C. Les programmes \verb+curl+
et \verb+tar+ seront appelés par \verb+exec+ ou dérivés.

\begin{solution}
  To perform the pipe, the main program forks into two sub programmes, one
  for each part of the pipe (\textit{Question for the reader:} can we use only
  one fork?). The main difficulty is to modify adequately file descriptors so
  that the download child writes the data where the decompressing child reads.
  \inputminted{C}{download.c}
\end{solution}

\titledquestion{La fonction d'Hénon, partie I}
On va calculer l'orbite d'un système dynamique de dimension 2. La fonction
d'Hénon est définie par le système
\begin{equation}
  \label{eq:1}
  H_{a,b} = \left\{
    \begin{aligned}
      x_{n+1} &= a - by_{n} - x_{n}^{2} \\
      y_{n+1} &= x_{n}.
    \end{aligned}
    \right.
\end{equation}
On utilisera un processus pour calculer la suite \( (x_{n})_{n} \) et un autre
processus pour calculer la suite \( (y_{n})_{n} \). Les processus échangeront
leurs données via un (ou des) tube(s). Pour éviter d'avoir à mettre en place une
synchronisation entre les processus, on pourra utiliser des pauses
\verb|sleep(1)|.

\begin{solution}
  \inputminted{C}{henon/henon_simple.c}
\end{solution}

Comment mettre en place une synchronisation par signaux?
\begin{solution}
  Le père commence par transmettre le \textit{pid} des enfants entre eux, pour
  qu'ils puissent s'échanger des signaux. Ensuite on peut utiliser un
  \verb|sigwait| bien placé.
  \inputminted{C}{henon/henon_sync_signal.c}
\end{solution}

Par la suite, on crééra en plus un processus dédié à la sortie: ce processus
doit écrire des lignes sous la forme \texttt{0.3415 1.2451} où le premier nombre
est $x_{n}$ et le deuxième $y_{n}$ dans un fichier \texttt{henon.dat}.

On pourra tracer la fonction avec la commande \texttt{gnuplot henon.p} après
avoir téléchargé le script
\url{http://www.lsv.fr/~hondet/resources/archos/henon.p}. Le fichier
\texttt{henon.dat} doit être dans le même dossier que \texttt{henon.p}.

Pour \( a = 1.4 \) et \( b = -0.3 \), la fonction est chaotique (au sens de
Devaney), c'est-à-dire,
\begin{itemize}
  \item la fonction est sensiblement dépendante des conditions initiales (effet
    papillon, imprédictabilité);
  \item la fonction est topologiquement transitive (indécomposabilité);
  \item les points périodiques sont denses dans le domaine de définition de
    $H_{a, b}$ (régularité).
\end{itemize}
et possède un attracteur étrange.

Pour plus d'informations sur les systèmes dynamiques, voir \textit{An
  introduction to Chaotic Dynamical Systems}, Robert~L.~Devaney, Westview.

\begin{solution}
  \inputminted[tabsize=2]{C}{henon/henon.c}
\end{solution}

\titledquestion{Coquille vide}
Vous trouverez le squelette de base du code C que l'on va utiliser pour recoder
un shell. Pour compiler le projet, utilisez la commande \mintinline{bash}{make}.
Par défault, le shell ne peut pas faire grand chose. On va essayer de le
compléter pas à pas. On va d'abord s'intéresser à la fonction
\mintinline{C}{execute}:

Le cas de base, correspond au cas \mintinline{C}{C_PLAIN}. Donner un exemple de
commande qui une fois parsée retourne un objet \mintinline{C}{cmd} tel que
\mintinline{C}{cmd->type == C_PLAIN}. En utilisant \mintinline{C}{ps}, observer
ce qui se passe dans un terminal lorsque vous lancez une commande.

Pour le moment, toute commande de base sera tout simplement exécutée. Pour
exécuter une commande, la librairie \texttt{glibc} offre un panel de fonctions
dont on peut avoir un aperçu en utilisant la commande

\begin{minted}{bash}
  man exec
\end{minted}

Selon vous, quelle fonction serait la plus appropriée dans notre cas
(justifier)? En utilisant toutes ces observations, remplir le trou
\mintinline{C}{C_PLAIN}.

Quel est le symbol pour l'opérateur de séquence en bash? Donner un exemple de
commande où la séquence se comporte différemment de l'opérateur \textit{et}
logique.

Implémenter le cas \mintinline{C}{C_SEQ}.

Implémenter les cas \mintinline{C}{C_AND} et \mintinline{C}{C_OR}.

Il est possible en bash d'écrire une commande de la forme
\begin{mintinline}{bash}
  (cmd1 \&\& cmd2 | cmd3 ...) 2>/dev/null
\end{mintinline}

Quel est le rôle des parenthèses dans la commande ci-dessus? Donner un exemple
d'une commande qui utilise (de façon non triviale) ces parenthèses.
Implémenter le cas \mintinline{C}{C_VOID}.

Que se passe-t-il si vous faites \texttt{CTRL+C} dans notre shell? Proposer une
solution pour récupérer la main après que l'utilisateur a entré \texttt{CTRL+C}.

Que se passe-t-il lorsque que vous lancez la commande
\begin{minted}{C}
  ls > dump
\end{minted}
Pour corriger ce problème, je vous invite à lire les pages de manuel
\mintinline{bash}{man stdin} et \mintinline{bash}{man dup}. En utilisant toutes
ces informations, implémenter la fonction \mintinline{C}{apply_redirections}
puis modifier votre implémentation pour que la commande ci-dessus se comporte
comme prévu.

Il nous reste finalement à implémenter le cas \mintinline{C}{C_PIPE}, pour cela
je vous invite à regarder le manuel de \mintinline{bash}{man pipe}. Donner un
exemple qui met en évidence pourquoi on ne peut pas simplement utiliser
\mintinline{bash}{dup2} pour réimplémenter le pipe? En utilisant la fonction
\mintinline{bash}{pipe}, réimplémenter le cas \mintinline{C}{C_PIPE}.

À ce stade, nous avons implémenté un \textit{shell} très rudimentaire, cependant
il est possible de l'étendre de bien des manières. Voici quelques possibilités
d'extensions qui peuvent vous rapporter des points bonus:

\begin{itemize}
  \item Réimplémenter les commandes \mintinline{bash}{ls},
    \mintinline{bash}{cat} ou \mintinline{bash}{cd}
  \item Implémenter l'extension des wildcards: \mintinline{bash}{ls *.pdf}
  \item Implémenter les processus de fond via les commandes:
    \mintinline{bash}{jobs}, \mintinline{bash}{bg}, \mintinline{bash}{fg}, \dots
\end{itemize}

\end{questions}
\end{document}
