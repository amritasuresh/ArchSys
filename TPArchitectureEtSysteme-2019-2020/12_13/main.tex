\documentclass{exam}
\usepackage{../main}

\if\answers0\noprintanswers\fi
\if\answers1\printanswers\fi

\newcommand{\Cplx}{\ensuremath{\mathbb{C}}}

\title{TP \( n + 2 \): Fils d'exécution et élections}
\date{}
\begin{document}
\maketitle
\begin{questions}
  \titledquestion{Ensembles de Julia par itération inverse}
On peut approcher l'ensemble de Julia d'une fonction complexe en
itérant son inverse un nombre suffisant de fois. Étudions la
(classique) fonction, avec \( c \in \Cplx \),
\[
  \begin{aligned}
    Q_c : \Cplx &\to \Cplx\\
    z &\mapsto z^2 + c
  \end{aligned}
\]
On définit l'orbite \`a rebours comme
\( \{ f_c^{-n}(z_{0}); n \in \mathbb{N} \} \).  Pour la calculer, on
note que si \( z^2 + c = w \), alors \( z = \rho\exp(i\theta) \) avec
\( \rho = \sqrt{|w - c|} \) et
\( \theta = \frac{\vartheta}{2} + \delta \pi \) avec \( \delta \in \{
0, 1 \} \) et
\[ \vartheta = \arctan(\Im(w - c)/\Re(w - c)) +
  \begin{cases}
    0 & \text{si } \Re(w-c) > 0 \\
    \pi & \text{sinon}
  \end{cases}
\]

Le point initial \( w_0 \) n'a pas d'importance.  On se propose de
créer un fil d'exécution par orbite, chaque orbite ayant un point
initial différent.

On notera les points ou bien sur la sortie standard, ou bien dans un
fichier, chaque ligne étant de la forme \verb|x y|. Si les points sont notés
dans un fichier nommé \verb|julia.dat|, on peut tracer avec la commande
\verb|gnuplot julia.p| en supposant l'existence du script
\cref{fig:gnuplot-julia}.
\begin{figure}[h]
\begin{verbatim}
set terminal png size 500,500
set output 'julia.png'
set title 'Julia set'
plot 'julia.dat'
\end{verbatim}
  \caption{Script gnuplot pour tracer l'ensemble de Julia à partir d'un fichier
    de données \texttt{julia.dat}}\label{fig:gnuplot-julia}
\end{figure}
On n'oubliera pas d'importer \texttt{math.h} et
\texttt{complex.h} et de compiler avec \texttt{-lm} en \emph{fin de
  ligne de commande}. La commande suivant devrait convenir,
\verb|gcc julia.c -lpthread -lm|.

On pourra essayer de tracer l'ensemble de Julia avec les paramètres
\( c \in \{ -1, -0.4 - 0.6i, -1.5, -i, -0.8 + 0.4i, 0.5, 3, 1 + i, 2
\} \)

\begin{solution}
  \inputminted{C}{correc/julia.c}
\end{solution}

\titledquestion{Serveur de dates: exclusion mutuelle simple}
On souhaite mettre en \oe{}uvre une architecture client serveur (sans pour
autant utiliser l'API socket, on se contentera d'avoir des fils d'exécution
client et un fil serveur) dans laquelle le serveur est chargé de
fournir la date chaque fois que le client la demande.

Le serveur tourne en boucle infinie. Il attend une requête de la part des
clients. Une fois la requête reçue, il émet une chaîne de caractères contenant
la date et l'heure à disposition du client. À ce moment là, le serveur est
prêt à répondre à une autre requête.

La tâche cliente exécutera 50 fois une fonction réalisant:
\begin{itemize}
  \item émission d'une requête au serveur,
  \item récupération de la date envoyée au serveur,
  \item affichage numéroté de la date.
\end{itemize}
\begin{parts}
  \part{} Quelle information est partagée entre les threads? Quel moyen de
  communication sera utilisé?
  \part{} Quel moyen de synchronisation sera utilisé?
  \part{} Implémenter la solution avec plus d'un client.
\end{parts}

\titledquestion{Élection d'un meneur}
Un problème important dans la programmation concurrente est de choisir
un \emph{leader}, e.g.\ pour l'organisation et la coordination d'un
réseau. On part avec un nombre $n$ d'ordinateurs ou de processus (on
parlera des \emph{nœuds}) à priori identiques, à l'exception d'un
identifiant unique. En suivant un même protocole, tous les nœuds vont
s'accorder sur le choix d'un leader.

Il existe une grande variété de protocoles menant à ce résultat.  Nous
allons dans ce TP en étudier et en réaliser un: le protocole Dolev,
Klawe, and Rodeh qui se distingue par le faible nombre de messages
échangé pour l'élection: $\mathcal{O}(n\log n)$. Par comparaison, les
approches simples ont tendance à utiliser $\mathcal{O}(n^2)$ messages.
Le protocole a été presenté en cours et les diapos explicatives sont
aussi disponibles dans le répertoire du TD.\@

Le protocole part du principe que les nœuds sont organiseés en anneau,
chaque nœud envoie des messages à son voisin de droite.  Le répertoire
du TP contient un squelette qui créera $n$ processus (les nœuds) pour
$n$ donné; ces processus seront déjà équipés de pipes pour
communiquer. Votre tâche est de compléter le programme en réalisant le
protocole:

\begin{itemize}
\item Au départ, tous les nœuds sont \emph{actifs} et possèdent un
  identifiant unique.

\item Un nœud actif attend une petite période aléatoire (fonction
  \texttt{delay()}), puis envoie son identifiant à son voisin de
  droite. Ensuite, il attend les identifiants de ses \emph{deux} plus
  proches voisins actifs à sa gauche.  Il décidera ensuite s'il reste
  actif, devient passif, ou se déclare leader. Les conditions seront
  précisées dans la présentation. S'il reste actif, il répète le
  comportement présenté ci-dessus.

\item Un nœud \emph{passif} transmet simplement tous les messages
  reçus de gauche vers son voisin de droite. En plus, si le message
  déclare un leader, il affiche un message correspondant à l'écran.

\item Il y a trois types de messages échangés par les nœuds, tous dans
  le format (\emph{type},\emph{identifiant}).

  \begin{itemize}
  \item ``voisin'' (v): pour envoyer son identifiant vers le voisin de
    droite.
  \item ``prochain'' (p): un nœud qui a reçu l'identifiant de son
    voisin de gauche l'envoie vers son voisin de droite.
  \item ``gagnant'' (g): un nœud se déclare leader.
  \end{itemize}
\end{itemize}

Un code d'amorçage est disponible a l'adresse
\url{http://www.lsv.fr/~hondet/resources/archos/ring.tar.gz}.

\begin{parts}
  \part{} Complétez la fonction <<protocole>> du fichier
  <<ring-pipe.c>> qui implémente le protocole sus-décrit.
  \begin{solution}
    ring-pipe crée plusieurs processus reliés par des pipes.
    \inputminted{C}{correc/ring/ring-pipe.c}
  \end{solution}
  \part{} Testez votre code avec plusieurs valeurs de $n$. Observez si
  votre programme termine toujours et s'il y a toujours un seul
  leader.
  \begin{solution}
    Au besoin réduire la valeur de DELAY pour $n$ grand.  Pour $n>255$
    le code de sortie (obtenu par \texttt{wait} ne suffit plus pour
    stocker l'identifiant du gagnant.
  \end{solution}
  \part{} La version réseau du programme, <<ring-net.c>> instancie un
  seul n\oe{}ud par exécution. Vous pouvez recopier votre fonction
  <<protocole>> depuis <<ring-pipe.c>>.
  Le programme prend 3 arguments~: le port d'écoute du n\oe{}ud, le nom du nœud voisin (nom de
  machine), le port de connexion au nœud voisin. Le programme crée un
  serveur dans un thread et attend la connexion d'un voisin. En
  parallèle, dans un autre thread, il tente une connexion sur son
  voisin (nom de machine et port passés en argument). Écrivez un
  script mettant en œuvre l'exécution de votre code sur 5 machines du
  département. Commencez par tester l'exécution de votre code
  localement (machine localhost en utilisant différents
  ports). Étendre progressivement la taille de l'anneau.
  \begin{solution}
    \inputminted{C}{correc/ring/ring-net.c}
  \end{solution}
  \part{} Créez un anneau avec vos voisins et l'étendre à toute la
  salle de TP.
\end{parts}

\titledquestion{Exclusion de Peterson}
Nous proposons d'implémenter l'algorithme d'exclusion mutuelle de Peterson décrit sur Wikipedia.
Le principe consiste à utiliser 3 variables \texttt{f0}, \texttt{f1} et \texttt{turn} pour gérer l'entrée dans la <<section critique>> du code.
La <<section critique>> est la section contenant le code à protéger des accès concurents.
Prouvez que algorithme assure l'exclusion mutuelle.
\begin{parts}
  \part{} En vous basant sur la page Wikipedia de cet algorithme :
\begin{itemize}
  \item modifiez la fonction \texttt{process0} et créez la fonction \texttt{process1} implémentant l'algorithme de Gary L. Peterson ;
  \textbf{Note:}  \texttt{process0}  et  \texttt{process1} impléméntent la même fonctionnalité mais pour les threads 0 et 1.
  \item Testez votre code et commentez.
 \end{itemize}
  \begin{solution}
    https://www-ssl.intel.com/content/dam/www/public/us/en/documents/manuals/64-ia-32-architectures-software-developer-system-programming-manual-325384.pdf Section 8-6 Vol. 3A 8.2.2 Memory Ordering in P6 and More Recent Processor Families

    ou
    \inputminted{C}{correc/peterson/peterson.c}
  \end{solution}
  \part{} En vous basant sur le code de \texttt{process0} et \texttt{process1} écrire la fonction \texttt{process} qui permettra l'invocation générique d'un thread.
\part{} Toujours en vous basant sur la page Wikipedia de l'algorithme de Peterson, modifiez votre code pour 4 processus.
\part{} Mêmes questions que précédemment pour l'algorithme de Dekker.
\end{parts}
\end{questions}

\end{document}
%%% Local Variables:
%%% mode: latex
%%% TeX-master: t
%%% End:
