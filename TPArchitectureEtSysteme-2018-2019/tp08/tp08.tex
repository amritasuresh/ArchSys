\documentclass[11pt]{article}
\usepackage[utf8]{inputenc}
\usepackage[T1]{fontenc}
\usepackage[francais]{babel}
\usepackage[francais]{layout}
%\usepackage{minted}
\selectlanguage{french}

% NE PAS CHANGER !!
\ifx \public \undefined \def\public{etudiants} \fi
\usepackage[\public]{tps}

% Numéro du TP
\newcommand{\numtd}{08}
% Titre du TP
\newcommand{\titretd}{Fork \& co}
\newcommand{\fork}{\mintinline{bash}{fork}}
\newcommand{\cmd}[1]{\verb|bash #1}
\graphicspath{{imgs/}}

\begin{document}

\entete{\numtd}{\titretd}

\begin{introduction}
Dans notre première séance de programmation système, on va principalement s'intéresser à la commande \verb|bash fork|.
\end{introduction}

\section{Prélude}

Pour écrire un programme C utilsant fork, il faut inclure les librairies suivantes dans vos programmes :

\begin{verbatim}
  #include <unistd.h>
\end{verbatim}

Vous aurez probablement par la suite besoin aussi des librairies suivantes :

\begin{verbatim}
  #include <sys/types.h>
  #include <sys/wait.h>
\end{verbatim}

\section{Forkons entre amis}

\paragraph{Un programme simple}

Le programme suivant montre une utilisation basique de fork{}.
\begin{verbatim}
#include <stdio.h>
#include <unistd.h>

int main() {
  printf("Bonjour,\n");
  fork();
  printf("J'ai fait un 'fork' !\n");
  return 0;
}
\end{verbatim}

\begin{itemize}
\item À votre avis, qu'affiche ce programme ? Tester pour vérifier le comportement attendu.
\item Écrire un programme qui étant donné une MACRO{C}{BOUND}, créer \(2^{BOUND}\) processus
\item Que fait la commande {pstree} ? Pouvez-vous observer le comportement de votre programme à l'aide de cette commande ? Pourquoi ?
\item Modifier très légèrement votre programme afin d'observer le comportement de votre programme à travers {pstree}.
\item Il est possible de tuer un processus à l'aide de la commande {kill}, comment ?
\item Observer l'effet de la commande {kill} sur votre programme à travera {pstree} ? Pourquoi à votre avis certains processus sont \textit{vraiment} tués et pas d'autres ?
\item En utilisant la fonction {wait}, corriger le problème identifié à la question précédente.
\item Modifier votre programme pour afficher les PIDS des processus crées. L'ordre est-il toujours le même ? Pourquoi ?
\item (Bonus) Expliquer le comportement de la commande suivante sans l'éxécuter (il est même fortement déconseillé de l'éxécuter) :
  \begin{verbatim}
    :(){ :|:& };:
  \end{verbatim}
\item (Bonus) Faire un programme paramétré par un entier \(n\) qui lance exactement \(n!\) processus
\end{itemize}

\paragraph{Tips and tricks}

\begin{itemize}
\item Quelle est la sortie attendue du programme suivant ? Éxécuter puis vérifier votre conjecture.
  \begin{verbatim}
#include <stdio.h>
#include <unistd.h>

int main() {
    printf("Je vais faire un 'fork'... ");
    int p = fork();
    if (p == 0) {
      printf("Je suis le fils...\n");
    } else {
      printf("Je suis le père...\n");
    }
    return 0;
}
\end{verbatim}
\item Que se passe t'il lorsqu'un processus fait un fork(), et que le père et le fils lisent sur l'entrée standard (avec getchar() par exemple) ? Décrivez votre méthodologie ainsi que les résultats trouvés.
\item  Écrivez un programme qui se comporte de la manière suivante :
  \begin{itemize}
  \item le processus initial créé un fils et se termine,
  \item le fils créé un fils (petit fils du processus initial) et se termine,
  \item le petit fils créé un fils (petit petit fils du processus initial) et se termine,
  \item etc...
  \end{itemize}
  Expliquez pourquoi le programme obtenu ne sature pas la table des processus.
\item Est-ce que le processus enfant peut tuer le processus père ? Comment ?
\end{itemize}

\paragraph{Attendre patiemment}

Expliquer les différences de comportement entre les trois programmes suivants (je vous invite à utiliser la commande {top} pour observer les différences) :
\begin{verbatim}
  int main() {
  while(1);
  return 0;
}
\end{verbatim}
\begin{verbatim}
#include <stdio.h>
int main() {
  getchar();
  return 0;
}
\end{verbatim}
\begin{verbatim}
#include <unistd.h>
int main() {
  while(1) {
    sleep(1);
  }
  return 0;
}
\end{verbatim}

\paragraph{Diviser pour mieux attendre}

Tester le programme suivant :
\begin{verbatim}
#include <stdio.h>
#include <unistd.h>
int main() {
    int p = getpid();
    printf("DÉBUT %i\n", p);
    unsigned n = 0x7fffffff;
    while(n--);
    printf("FIN %i\n", p);
    return 0;
}
\end{verbatim}

puis ajuster la valeur de mintinline{c}{n} pour que l'éxécution prenne environ entre \(5\) et \(10\) secondes.

\begin{itemize}
\item Que se passe-t-il si vous lancez deux fois ce programme en parallèles ? Quelle est la durée d'exécution totale de ces deux processus ? Refaire l'expérience pour \(4\) processus et prédire le résultat.
\item Que se passe t'il si on remplace la boucle mintinline{c}{while(n--);} par un mintinline{c}{sleep(5);} ?
\end{itemize}

\section{Calculons}

Dans cet exercice on  étudie l'utilisation et  l'évaluation du code de sortie d’un processus. Normalement, le code de sortie est utilisé pour indiquer par exemple, la résussite ou non d’un programme.
Ici, on s’en sert pour communiquer les résultats d’un calcul. (Remarque : le code de sortie doit
entre entre \(0\) et \(255\), donc ce n'est pas idéal pour ce genre de chose (c'est juste un exemple !).
Pages man utiles : {fork(2)}, {wait(2)}. Utilisez {make} pour compiler.
\begin{itemize}
\item On commence avec un exemple simple \texttt{simple.c} qui calcule la somme de deux entiers.
  Complétez le programme tel que le fils utilise {exit} pour communiquer la somme au père et que le père reçoit cette valeur par {wait}.
\item Application plus complexe : Télécharger l'archive \texttt{calc.zip} qui contient un calculateur simple qui évalue des expressions avec des entiers naturels, addition, multiplication et subtraction.
Dans l'état actuel le programme convertit l'expression vers un arbre, puis il traverse les noeuds de l’arbre un par un pour calculer les valeurs de toutes les sous-expressions. Votre tâche est de permettre au programme de calculer les sous-expressions en parallèle. Lorsque le programme évalue une sous-expression, il devrait lancer deux processus fils qui  évaluent  leurs  sous-expressions  et  qui  on  communiquent  le  résultat  au  père  par leurs codes de sortie. Le père attend les deux résultats et applique l'opération correspondante pour obtenir son propre résultat. Il suffit de modifier la fonction
compute dans main.
\end{itemize}

\section{Ping-Pong}

Écriver un programme qui créer un processus fils et qui à le comportement suivant : le père envoie un par un, \(1\) bit d'information au processus fils. Ce dernier doit afficher la suite de bits émise par son père \textbf{dans l'ordre}. Pour se faire, il vous est demandé d'utiliser les signaux. La difficulté de cet exercice est que l'ordre n'est pas garanti : un signal \(A\) émit avant le signal \(B\) par le processus père peut être reçu dans l'ordre inverse par le processus fils.

\end{document}
