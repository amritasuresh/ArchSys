\documentclass[11pt]{article}
\usepackage[utf8]{inputenc}
\usepackage[T1]{fontenc}
\usepackage[francais]{babel}
\usepackage[francais]{layout}
\selectlanguage{french}

% NE PAS CHANGER !!
\ifx \public \undefined \def\public{etudiants} \fi
\usepackage[\public]{tps}

% Numéro du TP
\newcommand{\numtd}{11}
% Titre du TP
\newcommand{\titretd}{Élection d'un leader}

\graphicspath{{imgs/}}

\begin{document}

\entete{\numtd}{\titretd}

Page web du cours :                                                                                                                                                                                                                                            
\begin{center}
\url{http://www.lsv.fr/~schwoon/enseignement/systemes/ws1718/}.
\end{center}

Sources du TP disponibles dans le répertoire
<<\textasciitilde{}fhh/share/tp/tp11>> du département informatique.

Un problème important dans la programmation concurrente est de choisir un
\emph{leader}, p.ex.\ pour l'organisation et la coordination d'un réseau. On
part avec un nombre $n$ d'ordinateurs ou de processus (on parlera des
\emph{nœuds}) à priori identiques, à l'exception d'un identifiant unique. En
suivant un même protocole, tous les nœuds vont s'accorder sur le choix d'un
leader.

Il existe une grande variété de protocoles menant à ce résultat.  Nous allons
dans ce TP en étudier et en réaliser un : le protocole Dolev, Klawe, and Rodeh
qui se distingue par le faible nombre de messages échangé pour l'élection :
$\mathcal{O}(n\log n)$. Par comparaison, les approches simples ont tendance à
utiliser $\mathcal{O}(n^2)$ messages.  Le protocole a été presenté en cours et
les diapos explicatives sont aussi disponibles dans le répertoire du TD.

Le protocole part du principe que les nœuds sont organiseés en anneau, chaque
nœud envoie des messages à son voisin de droite.  Le répertoire du TP contient
un squelette qui créera $n$ processus (les nœuds) pour $n$ donné ; ces processus
seront déjà équipés de pipes pour communiquer. Votre tâche est de compléter le
programme en réalisant le protocole :

\begin{itemize}
\item Au départ, tous les nœuds sont \emph{actifs} et possèdent un identifiant
unique.

\item Un nœud actif attend une petite période aléatoire (fonction
\texttt{delay()}), puis envoie son identifiant à son voisin de droite. Ensuite,
il attend les identifiants de ses \emph{deux} plus proches voisins actifs à sa
gauche.  Il décidera ensuite s'il reste actif, devient passif, ou se déclare
leader. Les conditions seront précisées dans la présentation. S'il reste
actif, il répète le comportement présenté ci-dessus.

\item Un nœud \emph{passif} transmet simplement tous les messages reçus de
gauche vers son voisin de droite. En plus, si le message déclare un leader, il
affiche un message correspondant à l'écran.

\item Il y a trois types de messages échangés par les nœuds, tous dans le
format (\emph{type},\emph{identifiant}).

\begin{itemize}
  \item ``voisin'' (v): pour envoyer son identifiant vers le voisin de droite.
  \item ``prochain'' (p): un nœud qui a reçu l'identifiant de son voisin de
gauche l'envoie vers son voisin de droite.
  \item ``gagnant'' (g): un nœud se déclare leader.
\end{itemize}
\end{itemize}

\begin{enumerate}
 \item Complétez la fonction <<protocole>> du fichier <<ring-pipe.c>> qui
implémente le protocole sus-décrit. 
 \begin{solution}
 ring-pipe crée plusieurs processus reliés par des pipes.
 \end{solution}
 \item Testez votre code avec plusieurs valeurs de $n$. Observez si votre
programme termine toujours et s'il y a toujours un seul leader.
 \begin{solution}
 Au besoin réduire la valeur de DELAY pour $n$ grand.
 Pour $n>255$ le code de sortie (obtenu par \texttt{wait} ne suffit plus pour 
 stocker l'identifiant du gagnant.
 \end{solution}
 \item La version réseau du programme, <<ring-net.c>> instancie un seul
n\oe{}ud par exécution. Vous pouvez recopier votre fonction <<protocole>>
depuis <<ring-pipe.c>>. Le programme prend 3 arguments : le port d'écoute du
n\oe{}ud, le nom du nœud voisin (nom de machine), le port de connexion au nœud
voisin. Le programme crée un serveur dans un thread et attend la connexion d'un
voisin. En parallèle, dans un autre thread, il tente une connexion sur son
voisin (nom de machine et port passés en argument). Écrivez un script mettant
en œuvre l'exécution de votre code sur 5 machines du département. Commencez par
tester l'exécution de votre code localement (machine localhost en utilisant
différents ports). Étendre progressivement la taille de l'anneau.
 \item Créez un anneau avec vos voisins et l'étendre à toute la salle de TP.
\end{enumerate}

\end{document}
