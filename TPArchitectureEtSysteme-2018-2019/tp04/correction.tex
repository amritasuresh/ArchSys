\documentclass{article}

\usepackage[utf8]{inputenc}
\usepackage{amsmath}
\usepackage{amssymb}
\usepackage{minted}

\begin{document}

Voici une correction des exercices \(3\) et \(4\) suite à mes explications pas forcéments claires vendredi dernier.

\section{Opérations binaires}

Il me semblait important que vous expérimentiez les questions en programmant, plutôt que de tout faire sur papier. Pour la correction, je choisirai de le faire en C, mais il est possible de le faire en OCaml ou bien en python. Cependant, en python, il est plus difficile de s'assurer que les entiers que l'on manipule sont bien sur \(32\) ou \(64\) bits\footnote{D'ailleurs, un \mintinline{C}{int} en C n'est pas forcément stocké sur \(32\) bits !}.

\subsection{Question 1}

Pour cette question, on pouvait utiliser n'importe quel entier, cela n'avait pas d'importance. Voici le code C qui implémente la fonction voulue :

\begin{minted}{C}
#include<stdio.h>

int question1(int n) {
  return n & (n - 1);
}

int main(int argc, char* argv[]) {
  for(int i = 0 ; i < 10 ; i++) {
    printf("%d:%d\n", i, question1(i));
  }
  return 0;
}
\end{minted}

ce qui affiche

\begin{minted}{bash}
0:0
1:0
2:0
3:2
4:0
5:4
6:4
7:6
8:0
9:8
\end{minted}

On remarque facilement que la sortie est paire. Cela peut s'expliquer car la parité de \(n\) et \(n-1\) n'est jamais la même. Et donc le bit le moins significiatif (LSB) de \mintinline{C}{n & n-1} est toujours \(0\). Afin d'avoir une meilleure intuition de ce qui se passe, on va afficher la représentation binaire de nos entiers. Malheureusement, la fonction \mintinline{C}{printf} de C ne sait pas faire ça automatiquement pour nous. On va donc devoir créer notre propre fonction. Pour simplifier les choses, je vais aussi seulement afficher les \(8\) premiers bits. La fonction n'est pas très compliquée : pour afficher le bit \(i\), on créer un masque (un entier) dont la représentation binaire contient des \(0\) partout sauf au bit \(i\) que l'on souhaite afficher. Puis on applique un \textit{et logique} avec notre nombre afin de récupérer ce bit. Cependant, la valeur que l'on obtient n'est pas \(1\) ou \(0\) mais \(2^i\) ou \(0\). On rédecale notre bit à droite pour afficher \(1\) ou \(0\). En C, cela se traduit donc par \mintinline{C}{(n & (1 << i)) >> i}. Comme on affiche de gauche à droite, et donc le bit le plus significatif en premier (MSB), on fera notre parcours de boucle dans l'autre sens. Ce qui donne in fine :
\begin{minted}{C}
void printb(int n) {
  for(int i = 7 ; i >= 0 ; i--) {
    printf("%d", (n & (1 << i)) >> i);
  }
}

int main(int argc, char* argv[]) {
  for(int i = 0 ; i < 10 ; i++) {
    printf("%d:%d\n", i, question1(i));
    printb(i);
    printf(":");
    printb(question1(i));
    printf("\n");
  }
  return 0;
}
\end{minted}

ce qui affiche

\begin{minted}{bash}
0:0
00000000:00000000
1:0
00000001:00000000
2:0
00000010:00000000
3:2
00000011:00000010
4:0
00000100:00000000
5:4
00000101:00000100
6:4
00000110:00000100
7:6
00000111:00000110
8:0
00001000:00000000
9:8
00001001:00001000
\end{minted}

En regardant la représentation binaire, la solution saute aux yeux : le bit à \(1\) le moins significatif est supprimé (enfin sauf pour \(0\)! Mais alors, pourquoi ?

Si \(n\) est impair, c'est évident. Si \(n\) est paire (et différent de \(0\)), alors le nombre peut s'écrire comme \((b_1b_2b_3b_4100\dots 000)_2\) et donc \(n-1\) vaut \((b_1b_2b_3b_4011\dots 111)_2\). Le fait de faire un \(\&\) logique entre ces deux nombres supprime donc bien le bit le moins significatif qui vaut \(1\).

\subsection{Question 2}

Cette fois on s'intéresse à la fonction suivante :
\begin{minted}{C}
int question2(int n) {
  int c;
  for (c = 0; n != 0; n &= (n-1)) c++;
  return c;
}
\end{minted}

L'affichage nous donne cette fois
\begin{minted}{bash}
0:0
00000000:00000000
1:1
00000001:00000001
2:1
00000010:00000001
3:2
00000011:00000010
4:1
00000100:00000001
5:2
00000101:00000010
6:2
00000110:00000010
7:3
00000111:00000011
8:1
00001000:00000001
9:2
00001001:00000010
\end{minted}

On constate alors que la fonction compte le nombre de bits à \(1\) dans la représentation binaire du nombre de départ. En effet, d'après la question précédente, chaque tour de boucle va supprimer un \(1\) dans la représentation binaire. On fait donc autant de passage dans la boucle qu'il y a de \(1\) dans la représentation binaire de ce nombre.
\end{document}
