\documentclass[11pt]{article}
\usepackage[utf8]{inputenc}
\usepackage[T1]{fontenc}
\usepackage[francais]{babel}
\usepackage[francais]{layout}
\selectlanguage{french}

% NE PAS CHANGER !!
\ifx \public \undefined \def\public{etudiants} \fi
\usepackage[\public]{tps}

% Numéro du TP
\newcommand{\numtd}{10}
% Titre du TP
\newcommand{\titretd}{Les threads}

\graphicspath{{imgs/}}

\begin{document}

\entete{\numtd}{\titretd}

Page web du cours :                                                                                                                                                                                                                                            
\begin{center}
\url{http://www.lsv.ens-cachan.fr/~schwoon/enseignement/systemes/ws1718/}.
\end{center}

Sources du TP disponibles dans le répertoire <<\textasciitilde{}fhh/share/tp/tp10>> du département informatique.

Pensez à utiliser \texttt{man} pour découvrir les détails
des appels système et des commandes dans le shell.

\section{Sémaphores}

Lors du TP09, nous avons étudié et implémenté l'algorithme d'exclusion mutuelle de Peterson et constaté les problèmes liés aux optimisations des processeurs récents.

Les \emph{sémaphores} représentent une solution pour résoudre le
problème des accès concurrents aux variables. Un sémaphore est
une strucutre de données utilisée pour la synchronisation et pour
assurer un accès coordonné aux ressources partagés. En C,
les sémaphores sont définis par le fichier
\texttt{semaphore.h}. Les fonctions les plus importantes pour nos
objectifs seront les suivantes :
%
\begin{itemize}
\item
  \verb|int sem_init(sem_t *sem, int pshared, unsigned int value);|\\ 
  Initialise \verb|sem| avec la valeur donnée ce qui donne le nombre
  de créneaux disponibles pour accéder à une ressource au même temps.
  Pour nos objectifs, \verb|pshared| devrait être \verb|NULL|.
\item \verb|int sem_wait(sem_t *sem);|\\ Si le sémaphore a la valeur 0
  cette fonction attend qu'un créneau devienne disponible. Sinon,
  elle décroît sa valeur immédiatement.
\item \verb|int sem_post(sem_t *sem);|\\ Augmente la valeur de la
  sémaphore (un créneau devient disponible).
\end{itemize}
%
étudiez les pages \verb|man| pour en savoir plus.

\begin{enumerate}
\item[(a)] Utilisez les sémaphores pour remplacer l'algorithme de Peterson et réparer le programme \texttt{counter.c}.
\end{enumerate}

Enfin, considérons le programme suivant (\texttt{order.c}) :
\begin{verbatim}
#include <pthread.h>
#include <semaphore.h>
#include <stdio.h>
#include <stdlib.h>

void* first(void* data)  { printf("First\n");  }
void* second(void* data) { printf("Second\n"); }
void* third(void* data)  { printf("Third\n");  }

int main () {
  pthread_t t1, t2, t3;
  
  pthread_create(&t3, NULL, third, NULL);
  pthread_create(&t2, NULL, second, NULL);
  pthread_create(&t1, NULL, first, NULL);

  // wait for all threads
  pthread_join(t1, NULL);  pthread_join(t2, NULL); pthread_join(t3, NULL);
}
\end{verbatim}
Modifiez le programme pour assurer que le programme donne toujours la
 sortie suivant :
\begin{verbatim}
  First
  Second
  Third
\end{verbatim}

\begin{enumerate}
\item[(b)]
 écrivez un programme en C avec deux threads. Le premier thread
 affiche les entiers pairs jusquà 100, le deuxième les impairs.
 Utilisez des sémaphores pour assurer le bon ordre des entiers.
\end{enumerate}

\section{Producer/consumer}

Un problème typique dans la programmation concurrente est celui du
\emph{producteur et consommateur} : un processus produit des données
que l'autre consomme. On considère le programme 
\texttt{procon.c}. Ici, le producteur obtient des caractères dans un
fichier et les envoie vers un espace partagé avec le processus consommateur
qui les affiche sur l'écran. Cependant, dans la version actuelle,
aucune synchronisation n'existe entre les deux.

Modifiez le programme tel qu'il affiche le contenu d'un fichier correctement.

\section{Mandelbrot}

Soit $c$ un nombre complexe. On considère la série
$z_0:=0$ et $z_{n+1}:=z_n^2+c$ pour $n\ge0$. L'\emph{ensemble de Mandelbrot}
est défini comme l'ensemble des valeurs $c$ telles que la série des $z_n$
est bornée. On sait que cela est le cas si $z_n$ ne sort jamais d'un
cercle de radius 2 autour de 0. Si jamais la série sort de ce cercle,
soit $m(c)$ le plus petit indice $n$ tel que c'est le cas.

Une application populaire pour $m(c)$ est de créer
de jolies images ; on associe l'écran avec un rectangle et chaque pixel
avec la valeur $c$ qui y correspond ; le pixel est ensuite peint avec
une couleur correspondant à $m(c)$.

La page web du cours contient un tel programme. Votre tâche consiste
à l'accélérer en lançant plusieurs threads en parallèle.
Les machines de la salle 411 sont équipées de plusieurs c\oe{}urs
(utilisez \texttt{cat /proc/cpuinfo} pour trouver plus d'information).
Chaque thread va donc travailler sur une partie différente de l'image.

Note: Compilez le programme avec \texttt{make}.

Note (2): Cet exercice ne nécessite pas de sémaphores.

Pages man utiles : pthreads(7), pthread\_create(3), pthread\_join(3).

\end{document}
