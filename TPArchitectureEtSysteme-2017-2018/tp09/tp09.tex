\documentclass[11pt]{article}
\usepackage[utf8]{inputenc}
\usepackage[T1]{fontenc}
\usepackage[francais]{babel}
\usepackage[francais]{layout}
\selectlanguage{french}

% NE PAS CHANGER !!
\ifx \public \undefined \def\public{etudiants} \fi
\usepackage[\public]{tps}

% Numéro du TP
\newcommand{\numtd}{09}
% Titre du TP
\newcommand{\titretd}{Programmation}

\graphicspath{{imgs/}}

\begin{document}

\entete{\numtd}{\titretd}

\section{Calculatrice}

Dans cet exercice on étudie l'utilisation et l'évaluation du code de sortie d'un processus.
Normalement, le code de sortie est utilisé pour indiquer par exemple la réussite ou non d'un programme.
Ici, nous nous en servirons pour communiquer les résultats d'un calcul.
Remarque : le code de sortie est compris entre 0 et 255, ce n'est pas idéal, mais il s'agit juste d'un exemple.
Références : fork(2), wait(2) et \texttt{make}

\begin{enumerate}
 \item Commençons par un exemple simple (\texttt{simple.c}) qui calcule la somme de deux entiers.
 Complétez le programme tel que le fils utilise \texttt{exit} pour communiquer la somme au père et que le père reçoive cette valeur par \texttt{wait} / \texttt{WEXITSTATUS}.

 \item Application plus complexe : la page web du cours contient un calculateur simple qui évalue des expressions avec des entiers naturels, addition, multiplication et soustraction.

Dans l'état actuel le programme convertit l'expression vers un arbre, puis il traverse les nœuds de l'arbre un par un pour calculer les valeurs de toutes les sous-expressions.
Votre tâche est de permettre au programme de calculer les sous-expressions \emph{en parallèle}.

Lorsque le programme évalue une sous-expression, il devra lancer deux processus fils qui évalueront leurs sous-expressions et qui en communiqueront le résultat au père par leurs codes de sortie.
Le père attend les deux résultats et applique l'opération (+,*,-) pour obtenir son propre résultat.
Il suffit de modifier la fonction \texttt{compute} dans \texttt{main}.

Remarque : Il est légèrement plus facile de réaliser la multiplication et l'addition que la subtraction -- pourquoi ?

\end{enumerate}

\section{Exclusion mutuelle}

Dans cet exercice nous illustrerons les problèmes d'accès concurent à une variable et nous nous initierons a la programmation des threads.

Cette fois, pas de code source de départ, le sujet vous aidera un peu mais vous écrirez votre code "from scratch" (ceci est une indication subliminale).

\begin{enumerate}
 \item Créez un code source composé :
 \begin{itemize}
  \item d'une constante $x$ supérieure à 10~000 ;
\begin{solution}
 Suggérer d'utiliser define
\end{solution}
  \item d'une fonction \texttt{process0} incrémentant $x$ fois une variable globale \texttt{cpt} ;
  \item d'un corps de programme instanciant deux threads, attendant le résultat de ces executions et l'affichant sur la sortie standard.
 \end{itemize}
 Avant de commencer à écrire votre code, quel est le résultat attendu ? Est-ce le résultat obtenu ? Pourquoi ?

 \item Algorithme de Peterson.

 Nous proposons de solutionner le problème précédent en implémentant l'algorithme d'exclusion mutuelle de Peterson décrit sur Wikipedia.
 Le principe consiste à utiliser 3 variables \texttt{f0}, \texttt{f1} et \texttt{turn} pour gérer l'entrée dans la <<section critique>> du code.
 La <<section critique>> est la section contenant le code à protéger des accès concurents.

 Prouvez que algorithme assure l'exclusion mutuelle.

 \item En vous basant sur la page Wikipedia de cet algorithme :
 \begin{itemize}
  \item modifiez la fonction \texttt{process0} et créez la fonction \texttt{process1} implémentant l'algorithme de Gary L. Peterson ;

\textbf{Note:}  \texttt{process0}  et  \texttt{process1} impléméntent la même fonctionnalité mais pour les threads 0 et 1.
  \item Testez votre code et commentez.
  \begin{solution}
    https://www-ssl.intel.com/content/dam/www/public/us/en/documents/manuals/64-ia-32-architectures-software-developer-system-programming-manual-325384.pdf Section 8-6 Vol. 3A 8.2.2 Memory Ordering in P6 and More Recent Processor Families
  \end{solution}
 \end{itemize}

 \item En vous basant sur le code de \texttt{process0} et \texttt{process1} écrire la fonction \texttt{process} qui permettra l'invocation générique d'un thread.

 \item Toujours en vous basant sur la page Wikipedia de l'algorithme de Peterson, modifiez votre code pour 4 processus.

 \item Mêmes questions que précédemment pour l'algorithme de Dekker.

\end{enumerate}



\end{document}
