\documentclass[11pt]{article}
\usepackage[utf8]{inputenc}
\usepackage[T1]{fontenc}
\usepackage[francais]{babel}
\usepackage[francais]{layout}
\selectlanguage{french}

% NE PAS CHANGER !!
\ifx \public \undefined \def\public{etudiants} \fi
\usepackage[\public]{tps}

% Numéro du TP
\newcommand{\numtd}{08}
% Titre du TP
\newcommand{\titretd}{Programmation d'un shell}

\graphicspath{{imgs/}}

\begin{document}

\entete{\numtd}{\titretd}

\begin{introduction}
L'objectif de cette séance est de créer votre propre shell.
Enfin, pour rendre cela possible dans une seule séance certains travaux
de programmation ennuyants ont déjà été faits, il suffit de remplir quelques
trous dans le code.
\end{introduction}

\section{Programmer un shell}

Vous trouverez un squelette
de code sous l'adresse suivante :

\url{http://www.lsv.fr/~schwoon/enseignement/systemes/ws1718/shell.zip}

Le code existant sait déjà accepter une commande, découper une ligne en
séparant les arguments, et d'exécuter une commande simple comme
\texttt{cat toto}. En plus, le code existant connaît le syntaxe pour
les redirections (\texttt{>}, \texttt{<}) et pour
enchaîner plusieurs commandes avec \texttt{;}, \texttt{||}, \verb+&&+
ou des tubes (\texttt{|}). Mais il ne sait pas, pour l'instant, comment
réaliser ces choses.

\subsection{Premiers essais}

Téléchargez le code et compilez-le avec \texttt{make}, puis exécutez le
shell. Vous verrez qu'une commande comme \texttt{ls} marche, mais qu'une
commande avec redirections ou enchaînement affiche un simple message d'erreur.
On termine le shell avec \verb-Ctrl+D-.

Regardez le fichier \texttt{global.h}. Il contient la définition d'une simple
structure de données qui représente le contenu d'une ligne de commande. Le
parser du shell remplit une telle structure quand l'utilisateur saisit une
commande. Quelques exemples :
\begin{itemize}
\item Commande simple (\texttt{cat toto}) : le champ \texttt{type} est
  \verb+C_PLAIN+, \texttt{args} contient un tableau avec deux
  éléments : \verb+{ "cat", "toto" }+. Ce qui est justement le format
  demandé par \verb+execvp+, par hasard\ldots
\item Commande avec redirection (\verb+cat toto > tata+) : comme avant,
  mais le champ \verb+output+ contient \verb+"tata"+. Les autres redirections
  sont dans \verb+append+, \verb+error+, et \verb+input+ qui représentent
  \verb+>>+, \verb+2>+ et \verb+<+.
\item Commande avec enchaînement (\verb+cat toto | wc+) : la structure devient
  une arborescence avec un nœud du type \verb+C_PIPE+ dans la racine et deux
  fils (champs \verb+left+, \verb+right+) représentant les commandes à gauche
  et à droite. Les autres enchaînements sont représentés par les types
  \verb+C_AND+ (\verb+&&+), \verb+C_OR+ (\verb+||+), \verb+C_SEQ+ (\verb+;+).
\item Enfin, une commande est du type \verb+C_VOID+ quand l'utilisateur a mis
  des parenthèses autour d'un groupe de commandes qui lui se trouve dans
  le fils \texttt{left}. Le nœud \verb+C_VOID+ peut pourtant avoir ses propres
  redirections.
\end{itemize}

Regarder maintenant le fichier \texttt{main.c} qui est normalement le seul
que vous aurez besoin de modifier. La fonction \verb+execute+ prend en charge
une structure et si elle est du type \verb+C_PLAIN+ sans redirection, elle
sait quoi faire et renvoie le code de sortie de cette commande.
À vous de compléter la fonction \ldots

\subsection{Les enchaînements}

Réalisez les enchaînements avec \verb+;+, \verb+||+ et \verb+&&+.
On rappelle que \verb+||+ exécute la deuxième commande si la code
de sortie de la première est non-zéro, et que \verb+&&+ celui-ci
doit être zéro.

\subsection{Redirections}

Réalisez les redirections. On rappelle que pour un processus,
le descripteur 0 est l'entrée standard, 1 la sortie standard
et 2 la sortie erreur. Quelques fonctions utiles (regardez les
pages man ou les transparents du cours pour plus d'informations) :
\begin{itemize}
\item \verb+open+ : ouvrir un fichier
\item \verb+dup2+ : recopier un descripteur vers un autre
\end{itemize}
Par exemple, pour rediriger l'entrée standard, il va falloir changer le
descripteur 0\ldots

\subsection{Tubes}

On s'occupe enfin des tubes. Pour un \verb+int fd[2]+ l'appel \verb+pipe(fd)+
crée deux descripteurs. Les données écrites dans \verb+fd[1]+ sont disponibles
pour lecture dans \verb+fd[0]+. Pour une commande comme \verb+cat toto | wc+
votre shell doit donc lancer deux fils qui partagent un tube et qui font
les bonnes manipulations sur leurs descripteurs avant de lancer les deux 
commandes. Il convient de fermer tout descripteur inutile dans le père
et les deux fils.

\subsection{Gérer Ctrl-C}

Essayez une commande comme \verb+cat+ sans arguments (qui attend simplement
que l'utilisateur saisisse des données pour les répéter). La combinaison
Ctrl-C interrompt cette commande mais, malheureusement, aussi le shell.
Faites en sorte que le shell ne soit plus terminé par Ctrl-C mais que
les processus fils le soient. Fonction utile : \texttt{signal}.

\subsection{Changer de dossier}

Malheureusement on ne peut pas se déplacer vers un autre dossier pour l'instant ;
la commande \verb+cd ..+ échoue. Réalisez une commande interne du shell
qui effectue un changement de dossier. Fonction utile : \texttt{chdir}.

\section{Base de données}

Cette exercice est destiné à ceux qui avancent vite. Le principe des pipes
peut aussi être utilisé de façon interactive. Ceci vous permet de faire
interagir votre programme avec d'autres programmes existants qui offrent
des fonctionnalités puissantes. On va démontrer cet effet en connectant
notre programme à une base de données.

\subsection{Premiers essais}

Téléchargez l'archive sous l'adresse

\url{http://www.lsv.fr/~schwoon/enseignement/systemes/ws1718/departement.zip}

Vous trouverez deux fichiers, un squelette de code (\texttt{departement.c})
et une base de données (\texttt{dep.sqlite}).
Dans votre terminal, lancez la commande \verb+sqlite3 dep.sqlite+.
Ceci vous propose une interface pour lancer des requêtes SQL. Sans entrer
dans les détails du langage SQL, essayez \verb+select * from dep;+.
Vous obtiendrez une liste de départements français (en métropole) avec 
leurs noms et chef-lieux. On peut être plus spéficique :
\verb+select nom,cheflieu from dep where nr='94';+ n'affiche que les
données pour le Val-de-Marne. Vous remarquerez que \verb+sqlite3+ accepte
des requêtes sur son entrée standard et affiche les réponses sur sa sortie
standard.

\subsection{Interfacer avec la base de données}

Utiliser une interface avec une base de données est très utile pour stocker
et récupérer les données d'un programme. (Par exemple, le serveur
pédagogique fonctionne de cette manière). L'idée, pour votre programme, est de lancer
un fils qui exécute un client de base de données (comme \verb+sqlite3+)
et de communiquer avec celui-ci en utilisant des tubes : un pour envoyer
des requêtes, l'autre pour récupérer les réponses. Le premier sera connecté
à l'entrée standard d'un processus sqlite, le deuxième à la sortie standard.

Regardez le programme \verb+departement.c+ et complétez-le. Le programme
est censé accepter un numéro de département et afficher ses données
(nom, cheflieu) sur l'écran. À vous d'ajouter les bonnes opérations
sur les tubes.

\end{document}
